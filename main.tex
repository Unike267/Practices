%%%%%%%%%%%%%%%%%%%%%%%%%%%%%%%%%%%%%%%%%
% Masters/Doctoral Thesis 
% LaTeX Template
% Version 2.5 (27/8/17)
%
% This template was downloaded from:
% http://www.LaTeXTemplates.com
%
% Version 2.x major modifications by:
% Vel (vel@latextemplates.com)
%
% This template is based on a template by:
% Steve Gunn (http://users.ecs.soton.ac.uk/srg/softwaretools/document/templates/)
% Sunil Patel (http://www.sunilpatel.co.uk/thesis-template/)
%
% Template license:
% CC BY-NC-SA 3.0 (http://creativecommons.org/licenses/by-nc-sa/3.0/)
%
%%%%%%%%%%%%%%%%%%%%%%%%%%%%%%%%%%%%%%%%%

%----------------------------------------------------------------------------------------
%	PACKAGES AND OTHER DOCUMENT CONFIGURATIONS
%----------------------------------------------------------------------------------------

\documentclass[
11pt, % The default document font size, options: 10pt, 11pt, 12pt
%oneside, % Two side (alternating margins) for binding by default, uncomment to switch to one side
spanish, % ngerman for German
singlespacing, % Single line spacing, alternatives: onehalfspacing or doublespacing
%draft, % Uncomment to enable draft mode (no pictures, no links, overfull hboxes indicated)
%nolistspacing, % If the document is onehalfspacing or doublespacing, uncomment this to set spacing in lists to single
%liststotoc, % Uncomment to add the list of figures/tables/etc to the table of contents
%toctotoc, % Uncomment to add the main table of contents to the table of contents
%parskip, % Uncomment to add space between paragraphs
%nohyperref, % Uncomment to not load the hyperref package
headsepline, % Uncomment to get a line under the header
%chapterinoneline, % Uncomment to place the chapter title next to the number on one line
%consistentlayout, % Uncomment to change the layout of the declaration, abstract and acknowledgements pages to match the default layout
]{MastersDoctoralThesis} % The class file specifying the document structure

\usepackage[utf8]{inputenc} % Required for inputting international characters
\usepackage[T1]{fontenc} % Output font encoding for international characters

\usepackage{mathpazo} % Use the Palatino font by default

\usepackage[backend=bibtex,style=ieee,natbib=true]{biblatex} % Use the bibtex backend with the authoryear citation style (which resembles APA)

\addbibresource{references.bib} % The filename of the bibliography

\usepackage[autostyle=true]{csquotes} % Required to generate language-dependent quotes in the bibliography
\usepackage[skip=4pt plus1pt, indent=15pt]{parskip}
\usepackage{pdfpages}
\usepackage{amsmath}
\usepackage[
    left = \flqq{},% 
    right = \frqq{},% 
    leftsub = \flq{},% 
    rightsub = \frq{} %
]{dirtytalk}
\usepackage{multirow}
\usepackage{diagbox}
\usepackage[chapter, newfloat]{minted} % Code listings, with syntax highlighting
\usepackage[font=small]{caption}
\usepackage{float}

\newenvironment{code}{\captionsetup{type=listing}}{}
\SetupFloatingEnvironment{listing}{name=Código}

%----------------------------------------------------------------------------------------
%	MARGIN SETTINGS
%----------------------------------------------------------------------------------------

\geometry{
	paper=a4paper, % Change to letterpaper for US letter
	inner=2.5cm, % Inner margin
	outer=3.8cm, % Outer margin
	bindingoffset=.5cm, % Binding offset
	top=1.5cm, % Top margin
	bottom=1.5cm, % Bottom margin
	%showframe, % Uncomment to show how the type block is set on the page
}

%----------------------------------------------------------------------------------------
%	THESIS INFORMATION
%----------------------------------------------------------------------------------------

\thesistitle{Integración y caracterización del rendimiento de unidades coprocesadoras en núcleo RISC-V para aplicaciones de IA} % Your thesis title, this is used in the title and abstract, print it elsewhere with \ttitle
\supervisor{Dr. Koldo \textsc{Basterretxea Oyarzabal}} % Your supervisor's name, this is used in the title page, print it elsewhere with \supname
\examiner{} % Your examiner's name, this is not currently used anywhere in the template, print it elsewhere with \examname
\degree{Master en Sistemas Electrónicos Avanzados} % Your degree name, this is used in the title page and abstract, print it elsewhere with \degreename
\author{Unai \textsc{Sainz Estebanez}} % Your name, this is used in the title page and abstract, print it elsewhere with \authorname
\addresses{} % Your address, this is not currently used anywhere in the template, print it elsewhere with \addressname

\subject{Electrónica Digital} % Your subject area, this is not currently used anywhere in the template, print it elsewhere with \subjectname
\keywords{} % Keywords for your thesis, this is not currently used anywhere in the template, print it elsewhere with \keywordnames
\university{\href{https://www.ehu.eus}{Universidad del País Vasco}} % Your university's name and URL, this is used in the title page and abstract, print it elsewhere with \univname
\department{} % Your department's name and URL, this is used in the title page and abstract, print it elsewhere with \deptname
\group{\href{https://www.ehu.eus/es/web/gded}{Grupo de Investigación de Diseño en Electrónica Digital}} % Your research group's name and URL, this is used in the title page, print it elsewhere with \groupname
\faculty{} % Your faculty's name and URL, this is used in the title page and abstract, print it elsewhere with \facname

\AtBeginDocument{
\hypersetup{pdftitle=\ttitle} % Set the PDF's title to your title
\hypersetup{pdfauthor=\authorname} % Set the PDF's author to your name
\hypersetup{pdfkeywords=\keywordnames} % Set the PDF's keywords to your keywords
}

\begin{document}

\renewcommand{\listtablename}{Índice de Tablas}
\renewcommand{\tablename}{Tabla}

\frontmatter % Use roman page numbering style (i, ii, iii, iv...) for the pre-content pages

\pagestyle{plain} % Default to the plain heading style until the thesis style is called for the body content

%----------------------------------------------------------------------------------------
%	TITLE PAGE
%----------------------------------------------------------------------------------------

\begin{titlepage}
\begin{center}

\vspace*{.06\textheight}
{\scshape\LARGE \univname\par}\vspace{1.5cm} % University name
\textsc{\Large Trabajo Fin de Máster}\\[0.5cm] % Thesis type

\HRule \\[0.4cm] % Horizontal line
{\huge \bfseries \ttitle\par}\vspace{0.4cm} % Thesis title
\HRule \\[1.5cm] % Horizontal line
 
\begin{minipage}[t]{0.4\textwidth}
\begin{flushleft} \large
\emph{Autor:}\\
\href{http://www.nuberoja.com}{\authorname} % Author name - remove the \href bracket to remove the link
\end{flushleft}
\end{minipage}
\begin{minipage}[t]{0.4\textwidth}
\begin{flushright} \large
\emph{Director:} \\
\href{https://www.ehu.eus/es/web/gded/koldo}{\supname} % Supervisor name - remove the \href bracket to remove the link  
\end{flushright}
\end{minipage}\\[3cm]
 
\vfill

\large \textit{El presente Trabajo Fin de Máster}\\[0.3cm] % University requirement text
\textit{se ha realizado en el}\\[0.4cm]
\groupname\\[2cm] % Research group name and department name
 
\vfill

{\large \today}\\[4cm] % Date
%\includegraphics{Logo} % University/department logo - uncomment to place it
 
\vfill
\end{center}
\end{titlepage}

\let\cleardoublepage\clearpage 

%----------------------------------------------------------------------------------------
%	QUOTATION PAGE
%----------------------------------------------------------------------------------------

\vspace*{0.2\textheight}

\noindent\enquote{\itshape L'umanità è a un bivio. O torna a credere di avere una natura diversa rispetto alle macchine o sarà ridotta a macchina tra le macchine.}\bigbreak

\hfill Federico Faggin

%----------------------------------------------------------------------------------------
%	RESUMEN CASTELLANO
%----------------------------------------------------------------------------------------

\begin{Resumen}
\addchaptertocentry{\resumenname} % Add the abstract to the table of contents
Aqui iria el resumen propio en castellano\ldots
\end{Resumen}

%----------------------------------------------------------------------------------------
%	RESUMEN EUSKERA
%----------------------------------------------------------------------------------------

\begin{Laburpena}
\addchaptertocentry{\resumennameeus} % Add the abstract to the table of contents
Aqui iria el resumen propio en euskera\ldots
\end{Laburpena}

%----------------------------------------------------------------------------------------
%	RESUMEN INGLES
%----------------------------------------------------------------------------------------

\begin{Abstract}
\addchaptertocentry{\resumennamein} % Add the abstract to the table of contents
Aqui iria el resumen propio en ingles\ldots
\end{Abstract}

%----------------------------------------------------------------------------------------
%	LIST OF CONTENTS/FIGURES/TABLES PAGES
%----------------------------------------------------------------------------------------

\tableofcontents % Prints the main table of contents
\addchaptertocentry{\contentsname}  % Add the table of contents to the table of contents
\listoffigures % Prints the list of figures
\addchaptertocentry{\listfigurename} % Add the list of figures to the table of contents
\listoftables % Prints the list of tables
\addchaptertocentry{\listtablename} % Add the list of tables to the table of contents
%----------------------------------------------------------------------------------------
%	ABBREVIATIONS
%----------------------------------------------------------------------------------------

\begin{abbreviations}{ll} % Include a list of abbreviations (a table of two columns)
\addchaptertocentry{\abbrevname} % Add the list of abbreviations to the table of contents
\textbf{IA} & \textbf{I}nteligencia \textbf{A}rtificial\\
\textbf{AI} & \textbf{A}rtificial \textbf{I}ntelligence\\
\textbf{FPGA} & \textbf{F}ield \textbf{P}rogrammable \textbf{G}ate \textbf{A}rrays\\
\textbf{ALU} & \textbf{A}rithmetic \textbf{L}ogic \textbf{U}nit\\
\textbf{VHDL} & \textbf{VHSIC} \textbf{H}ardware \textbf{D}escription \textbf{L}anguage\\
\textbf{VHSIC} & \textbf{V}ery \textbf{H}igh \textbf{S}peed \textbf{I}ntegrated \textbf{C}ircuit\\
\textbf{CI} & \textbf{C}ontinuous \textbf{I}ntegration\\
\textbf{HDL} & \textbf{H}ardware  \textbf{D}escription \textbf{L}anguage\\
\textbf{ISA} & \textbf{I}nstruction  \textbf{S}et \textbf{A}rchitecture\\
\textbf{CPU} & \textbf{C}entral  \textbf{P}rocessing \textbf{U}nit\\ 
\textbf{ESA} & \textbf{E}uropean  \textbf{S}pace \textbf{A}gency\\ 
\textbf{SPARC} & \textbf{S}calable  \textbf{P}rocessor \textbf{ARC}hitecture\\ 
\textbf{POWER} & \textbf{P}erformance \textbf{O}ptimization \textbf{W}ith \textbf{E}nhanced \textbf{R}ISC\\   
\textbf{SoC} & \textbf{S}ystem  \textbf{o}n a \textbf{C}hip\\ 
\textbf{CFU} & \textbf{C}ustom  \textbf{F}unction \textbf{U}nit\\ 
\textbf{CXU} & \textbf{C}omposable  e\textbf{X}tension \textbf{U}nit\\ 
\textbf{AXI} & \textbf{A}dvanced  e\textbf{X}tensible \textbf{I}nterface\\ 
\textbf{SLINK} & \textbf{S}tream   \textbf{LINK}  Interface\\ 
\textbf{XBUS} & Processor-E\textbf{X}ternal  \textbf{BUS}  Interface\\ 
\textbf{CFS} & \textbf{C}ustom  \textbf{F}unction \textbf{S}ubsystem\\ 
\textbf{FIFO} & \textbf{F}irst  \textbf{I}n \textbf{F}irst \textbf{O}ut\\ 
\textbf{CRI} & \textbf{C}entered \textbf{R}ecursive \textbf{I}nterpolation\\ 
\textbf{MLO} & \textbf{M}odified \textbf{L}attice \textbf{O}perators\\ 
\textbf{RNA} & \textbf{R}edes \textbf{N}euronales \textbf{A}rtificiales\\ 
\textbf{FA} & \textbf{F}unción de \textbf{A}ctivación\\ 
\textbf{ReLU} & \textbf{Re}ctified \textbf{L}inear \textbf{U}nit\\ 
\textbf{FOSS} & \textbf{F}ree and \textbf{O}pen \textbf{S}ource \textbf{S}oftware\\
\textbf{ODS} & \textbf{O}bjetivos de \textbf{D}esarrollo \textbf{S}ostenible\\ 
\textbf{MAC} & \textbf{M}ultiply–\textbf{AC}cumulate\\ 
\textbf{CNN} & \textbf{C}onvolutional \textbf{N}eural \textbf{N}etwork\\ 
\textbf{DRAM} & \textbf{D}ynamic \textbf{R}andom \textbf{A}ccess \textbf{M}emory\\ 
\textbf{PCI} & \textbf{P}eripheral \textbf{C}omponent \textbf{I}nterconnect\\ 
\textbf{SATA} & \textbf{S}erial \textbf{A}dvanced \textbf{T}echnology \textbf{A}ttachment\\ 
\textbf{EDA} & \textbf{E}lectronic \textbf{D}esign \textbf{A}utomation\\
\textbf{FLOS} & \textbf{F}ree/\textbf{L}ibre and/or \textbf{O}pen \textbf{S}ource\\
\textbf{FPGAIF} & \textbf{FPGA} \textbf{I}nterchange \textbf{F}ormat\\
\textbf{ASIC} & \textbf{A}pplication-\textbf{S}pecific \textbf{I}ntegrated \textbf{C}ircuit\\
\textbf{GCC} & \textbf{G}NU \textbf{C}ompiler \textbf{C}ollection\\
\textbf{ILA} & \textbf{I}ntegrated \textbf{L}ogic \textbf{A}nalyzer\\
\textbf{IMEM} & \textbf{I}nstruction \textbf{MEM}ory\\
  
\end{abbreviations}

%----------------------------------------------------------------------------------------
%	DEDICATION
%----------------------------------------------------------------------------------------

\dedicatory{For/Dedicated to/To my\ldots} 

%----------------------------------------------------------------------------------------
%	THESIS CONTENT - CHAPTERS
%----------------------------------------------------------------------------------------

\mainmatter % Begin numeric (1,2,3...) page numbering

\pagestyle{thesis} % Return the page headers back to the "thesis" style


\chapter{Memoria} % Título del capítulo

\label{Memoria} % etiqueta \ref{Memoria}

\section{Introducción}

Atendiendo al contexto socioeconómico actual es razonable señalar que el conjunto de conocimientos, tanto teóricos como tecnológicos, asociados al desarrollo de la IA (Inteligencia Artificial) suponen un ámbito científico estratégico. 
En este sentido, la investigación actúa como herramienta vertebradora de su progreso. 
De esta manera, se faculta a los ingenieros/as para materializar y perfeccionar las innovaciones que definirán su evolución. 

En términos generales, la IA está basada en modelos de redes neuronales.
Existen varias formas de enfocar la gestión computacional de estos modelos.
La tendencia general es emplear unidades de procesamiento gráfico en combinación con procesadores complejos administrados por sistemas operativos. 
De este modo, se consigue gestionar grandes volúmenes de datos, lo que conlleva un entrenamiento óptimo y en consecuencia un aprendizaje exitoso.
No obstante, la infraestructura necesaria para tal fin ocupa un tamaño considerable.
Además, se requiere de una gran cantidad de energía.
Hay ocasiones en las que el tamaño y/o la potencia consumida son factores determinantes, por ejemplo, en unidades autónomas con capacidad de operar aisladas como satélites, vehículos autónomos etc.
En estos casos la IA es necesaria para el desempeño de sus funciones, pero su procesamiento debe estar descentralizado en su propio hardware.
Este procesamiento de algoritmos de \textit{Machine Learning} a nivel local y sin necesidad de estar conectados a internet es lo que se conoce como IA en el borde \cite{IA-edge} (\textit{AI at the edge}) y surge en contraposición a los modelos conectados a la nube.
Además, cabe destacar que realizar por completo el creciente número de procesos de IA en la nube no sólo no es deseable, sino que es insostenible \cite{Efficient-Inference}. 
% Machine Learning -> aprendizaje máquina

Para llevar a cabo la inferencia de IA en el borde, se sugiere un enfoque de gestión computacional distribuido.
A este respecto, se plantea el uso de FPGAs.
Se propone implementar en estos dispositivos una arquitectura SoC heterogénea con un microcontrolador acoplado a coprocesadores de aplicación específica.
De esta manera, se consigue distribuir las tareas con objeto de ganar flexibilidad y lograr una mayor eficiencia.
Por un lado, el microcontrolador permite programabilidad y capacidad de gestionar los datos de entrada/salida. 
Por otro lado, los coprocesadores embebidos permiten realizar las operaciones asociadas a las redes neuronales con un alto rendimiento. 
Existen varios modos de acoplar estos dos elementos.
A lo largo del desarrollo de este trabajo, se realiza una caracterización del rendimiento de los diferentes modos de conexión que ofrece el microcontrolador RISC-V seleccionado.
Mediante esta caracterización se pretende concluir el método de acoplamiento más adecuado en términos de latencia.
Como se ha mencionado, la sugerencia de distribución computacional para llevar a cabo IA en el borde conlleva externalizar los cálculos asociados a las redes neuronales del micro a coprocesadores embebidos.
En lo referente a este trabajo, se comienza por evaluar un acelerador relativo al cálculo de funciones de activación, particularmente la función sigmoide.
%Se escoge está función ya que al contener exponenciales su cálculo supone numerosos ciclos de reloj si se efectúan íntegramente mediante un microcontrolador. 
Se escoge esta función ya que requiere de operaciones matemáticas que no son directas para la ALU.
Esto le supone al microcontrolador numerosos ciclos de reloj para calcularla. 
Por lo tanto, emplear un coprocesador embebido para acelerar este proceso resulta en un ahorro significativo del tiempo de computación.
Este acelerador, se integra al microcontrolador mediante el método previamente concluido como el más adecuado. 
Finalmente, se corroboran los beneficios de emplear este tipo de enfoque distribuido.

La investigación presentada en este trabajo ha sido desarrollada en el Grupo de Investigación de Diseño en Electrónica Digital (GDED).
El grupo está familiarizado con el diseño hardware, así como con su descripción empleando principalmente el lenguaje VHDL. 
Asimismo, fomenta el uso de herramientas FLOS, así como la metodología de integración continua (CI), que en conjunción con programas/plataformas de control de versiones, han sido esenciales para la gestión eficaz de todo el código empleado.
En definitiva, la colaboración con el grupo ha sido determinante para la correcta realización de este Trabajo Fin de Máster.

\section{Contexto}

A la hora de implementar una arquitectura heterogénea, es preciso que los elementos que la componen se caractericen por su flexibilidad.
Uno de los pilares fundamentales en lo referente a una solución distribuida, es la parte relativa al procesador por instrucciones.
En este sentido, hay ocasiones en las que se requiere modificar sus fuentes descriptivas, con objeto de personalizar su funcionamiento.
De este modo, se logra adaptar la CPU a la coyuntura objetivo, por ejemplo, integrando en ella los coprocesadores específicos necesarios.
En consecuencia, se consigue amoldar en un diseño Sistema-en-Chip (SoC) la resolución relativa al ámbito de aplicabilidad concreto. 
Para conseguir la plasticidad descrita, se requiere interactuar con los diseños mediante las descripciones conceptuales de los mismos.
A este respecto, se propone trabajar con las fuentes HDL de procesadores basados en una ISA libre de derechos.
Esta decisión no es de corte dogmática, sino que está fundamentada en las posibilidades que ella ofrece, tales como: la fabricación en silicio libre de regalías (\textit{royalty-free}) de soluciones SoC basados en estas ISAs, la difusión sin restricciones en plataformas públicas de las descripciones de CPUs y derivados, la publicación de comparativas de rendimiento etc.
Efectuando un repaso por las principales ISAs abiertas/libres, existe una relativamente moderna que destaca sobre el resto, esta es RISC-V \cite{waterman13}.
Nativa de Berkeley, esta ISA ha definido, en gran medida, el panorama actual de implementaciones de procesadores de hardware libre. 
De este modo, numerosos usuarios, así como organizaciones, han distribuido sus propias soluciones de CPUs y microcontroladores basados en esta ISA.
Como resultado, la oferta de procesadores \textit{soft-core} se ha multiplicado en los últimos años, disponiendo de una accesibilidad de estas descripciones nunca antes vista.
Además, en lo que respecta a las investigaciones  relativas a la aceleración de aplicaciones de IA mediante coprocesadores acoplados a CPUs RISC-V, se ha observado un aumento considerable en los últimos 3 años.
Adicionalmente, cabe destacar que la trascendencia del proyecto RISC-V ha sido tan significativa que la industria privada relacionada con el sector se ha visto obligada a ofertar sus propias soluciones basadas en esta ISA o incluso a abrir alguna de sus ISAs cerradas para acomodarse al nuevo paradigma actual.

\subsection{ISAs libres}

\label{isas-free}

%explicación ISA
A grandes rasgos, la CPU o núcleo del microcontrolador, es la unidad encargada de decodificar y ejecutar secuencialmente las instrucciones que previamente extrae de un programa almacenado en memoria.
%Previamente? secuencial donde ponerlo
Estas instrucciones codifican operaciones y dependen de la arquitectura que las procesa para aplicarlas.
En otras palabras, el mecanismo secuencial de extracción/decodificación/ejecución de instrucciones transforma sucesivamente una operación (aritmeticológica, de movimiento de datos etc.) de un plano conceptual, codificada en un programa, a un plano material, es decir, físicamente haciendo uso de los recursos electrónicos de la arquitectura digital. 
Es por ello que el set de instrucciones está íntimamente ligado a la arquitectura y definirlo condiciona de manera abstracta la topología de la misma. 
En este sentido, una arquitectura de conjunto de instrucciones (ISA) es una especificación de instrucciones que permite generar arquitecturas de procesadores a partir de ella.
Además, este concepto debe estar complementado con un soporte para compilar lenguajes de alto nivel, como por ejemplo C, que mediante las herramientas y librerías necesarias traduzcan su sintaxis a instrucciones para una ISA en concreto.

%Estos tres recursos, arquitectura digital, ISA y herramientas para compilar forman un proyecto de microcontrolador.

En el siglo pasado, el desarrollo de microarquitecturas estaba completamente monopolizado por el ámbito privado.
Esto se debía a que el esfuerzo de definir una ISA y dar soporte para la compilación de su set de instrucciones, además de verificar todo su ecosistema, era demasiado elevado para afrontarlo desde un punto de vista \textit{Open Source}.
Es por ello que a comienzos del siglo XXI eran pocos los proyectos de hardware libre de este tipo.
Caben mencionar los procesadores OpenRISC \cite{openrisc} basados en su propia ISA y desarrollados por la comunidad OpenCores \cite{opencores} y los procesadores OpenSPARC \cite{opensparc} y LEON \cite{leon} (proyecto referente de la ESA), basados en la ISA SPARC.
No obstante, entre mediados de los 2000 y de la década de 2010, han aumentado las especificaciones de ISAs libres de derechos que se han puesto a disposición de los usuarios.
Esto se ha debido principalmente a tres motivos.
El primero de ellos es porque las patentes han expirado.
Este es el caso de la ISA SuperH \cite{superh}.
Durante la crisis económica asiática de 1997 Hitachi se asoció con Mitsubishi y escindió su división de microcontroladores en una nueva compañía llamada Renesas \cite{new-comp}.
Esta compañía no heredó los ingenieros de Hitachi que habían desarrollado SuperH, por lo que con el tiempo Renesas perdió el interés por ella y expiraron sus patentes.
Un ejemplo de diseño de código abierto que utiliza este set de instrucciones es el procesador J-core \cite{j-core}.
El segundo motivo es porque se han dado las circunstancias materiales necesarias para llevar a cabo un proyecto de ISA abierto concebido desde el primer momento como \textit{Open Source}.
Este es el caso de RISC-V \cite{waterman13}.
El tercer motivo es porque a consecuencia de la enorme relevancia del proyecto anterior, las compañías intencionalmente han decidido abrir las especificaciones. 
Este es el caso de POWER ISA \cite{power}.
A principios de la década de 1990, IBM desarrolló una ISA llamada IBM POWER (\textit{Performance Optimization With Enhanced RISC}) para la familia de procesadores POWER.
Más tarde, a finales de los 90, esta arquitectura se abandonó y lo que se había convertido en PowerPC evolucionó hasta que en 2006 se estableció como POWER ISA. 
En 2019, IBM decidió hacer esta ISA \textit{Open Source} y desde entonces es mantenida por la fundación OpenPower \cite{open-power}.
Tres ejemplos de diseños de código abierto que utilizan este set de instrucciones son Microwatt \cite {gh:microwatt}, Chiselwatt \cite{gh:chiselwatt} y Libre-SOC \cite{libre-soc}.

\subsection{Ecosistema RISC-V}

\label{eco-risc}

En mayo de 2010, como parte del Laboratorio de Computación Paralela (Par Lab) de la universidad de Berkeley, el profesor Krste Asanović y los estudiantes de posgrado Yunsup Lee y Andrew Waterman iniciaron la especificación de un set de instrucciones denominado RISC-V.
El Par Lab fue un proyecto dirigido por David Patterson y subvencionado con capital privado proveniente de empresas como Intel y Microsoft, además de con fondos públicos del estado de California.
Cabe destacar que todos los proyectos del Par Lab eran de código abierto y utilizaban la licencia \textit{Berkeley Software Distribution} (BSD), incluido RISC-V.
El 13 de mayo de 2011 se publicó el primer manual del set de instrucciones de RISC-V bajo el nombre \textit{The RISC-V Instruction Set Manual, Volume I: Base User-Level ISA} \cite{manual-riscv}.
Ese mismo año se realizó el primer \textit{tapeout} de un chip RISC-V, producido en 28nm.
Durante los siguientes años se realizaron publicaciones y talleres.
A este respecto, una publicación interesante es la realizada en 2014 por Asanović y Patterson titulada \textit{Instruction Sets Should Be Free: The Case For RISC-V} \cite{paper-riscv} donde se plantean argumentos que defienden una ISA libre y abierta. 
En concreto, RISC-V pasó a ser de dominio público en el momento en el que se publicaron los informes técnicos que definen la ISA, aunque el texto de dichos informes está bajo una licencia de \textit{Creative Commons} para permitir su mejora por colaboradores externos.
Además, no se ha registrado ninguna patente relativa a RISC-V, puesto que esta ISA per se no representa ninguna tecnología novedosa.
En 2015, se creó la fundación RISC-V con el objetivo de construir una comunidad abierta e internacionalizar el proyecto.
En 2018, la fundación RISC-V anunció una colaboración conjunta con la Fundación Linux en términos de cooperación técnica y estratégica, lo que incluye apoyo operativo, gestión de miembros, contabilidad, divulgación comunitaria etc. 
En marzo de 2020, la Asociación Internacional RISC-V se constituyó en Suiza debido a factores geopolíticos externos que defienden la continuidad del modelo de propiedad intelectual.
El interés de la Asociación Internacional RISC-V es animar a organizaciones, particulares y entusiastas a hacer posible una nueva era de innovación en procesadores a través de la colaboración en estándares abiertos y \textit{Open Source}.

En este sentido, las plataformas de control de versiones, como GitHub y GitLab, cumplen un papel crucial a la hora de compartir entre los usuarios sus diseños de núcleos y microcontroladores SoC basados en esta ISA libre.
En particular, el primer proyecto de un procesador RISC-V se escribió en Chisel \cite{bachrach:2012:chisel}.
No obstante, en la actualidad hay multitud de proyectos en todo tipo de lenguajes de descripción de hardware.
Existe una lista de implementaciones RISC-V de código abierto \cite{marchid} donde se pueden consultar los proyectos \textit{Open Source} más relevantes.
Además, a cada uno de ellos se le asigna un identificador (ID).
Algunos de estos ejemplos son:

\begin{itemize}
    \item En Scala:
        \begin{itemize}
            \item Rocket Chip \cite{gh:rocket-chip} ID: 1
            \item VexRiscv \cite{gh:VexRiscv} ID: 33
        \end{itemize}
    \item En Verilog:
        \begin{itemize}
            \item PicoRV32 \cite{gh:picorv32} ID: none \footnote{A pesar de no tener ID se considera un proyecto \textit{Open Source} de gran relevancia.}
            \item Hummingbirdv2 E203 \cite{gh:e203-hbirdv2} ID: 26
            \item Steel Core \cite{gh:riscv-steel} ID: 24
            \item SERV \cite{gh:serv} ID: 18
        \end{itemize}
    \item En VHDL:
        \begin{itemize}
            \item NEORV32 \cite{gh:neorv32} ID: 19
            \item ORCA \cite{gh:ORCA-risc-v} ID: 7
            %\item Potato \cite{gh:potato} ID: none
        \end{itemize}
\end{itemize} 

\subsubsection{CFUs/CXUs}

Una de las finalidades de este proyecto de investigación es realizar una caracterización del rendimiento de los métodos de acoplamiento de coprocesadores en núcleos RISC-V.
A este respecto, un procesador basado en RISC-V no solo se limita a ofrecer la posibilidad de dar soporte para conexiones mapeadas en memoria o comunicaciones mediante interfaces \textit{stream}.
Un concepto destacable que posibilitan los núcleos basados en esta ISA es el de gestionar instrucciones personalizadas (\textit{custom}).
Haciendo uso de este concepto surge la oportunidad de realizar extensiones de instrucciones personalizadas y asociarlas con aceleradores embebidos.
%puede que retocar-Hecho!
Esta vinculación resulta muy ventajosa, ya que permite a la CPU resolver aplicaciones especificas decodificando instrucciones \textit{custom}.
De esta modo, se deriva la información directamente al acelerador embebido de manera similar a como lo haría una instrucción de suma con la ALU. 
Es decir, se consigue una integración completa del acelerador dentro del \textit{pipeline} de la CPU, lo que significa una forma muy eficiente de acoplar micro y coprocesador.
A priori, lo que resulta como una ventaja evidente puede conllevar algunos inconvenientes.
%Comas aqui?-Retocado!
Esto es debido a que las librerías de software que utilizan estas extensiones y los núcleos que las implementan son creados por diferentes organizaciones.
Comúnmente estas utilizan herramientas diferentes y sus desarrollos, aunque aislados son operativos, al integrarlos para genera un nuevo sistema podrían no funcionar.
Es por ello que se requiere de cierta estandarización que permita la reutilización robusta de las extensiones y librerías, además de proporcionar un modelo de programación uniforme para todas ellas.

En este sentido, en 2019, Google propuso el concepto \textit{Custom Function Unit} (CFU) \cite{prakash23} y lo integró en la CPU VexRiscv.
A partir de entonces esta extensión ha ganado popularidad y otras microarquitecturas RISC-V han añadido esta funcionalidad.
Derivado de este trabajo, existe un borrador que especifica la integración hardware-software de estas unidades de funcionalidades personalizadas \cite{CFU-draft}. 
En este texto se propone la idea de \textit{Composable eXtension Unit} (CXU), que plantea una generalización al concepto de CFU. 
%Y se anima a cualquiera a definir, desarrollar y usar CXUs y sus librerías CX con el objetivo de que coexistan un número ilimitado de ellas desarrolladas de forma independiente dentro de una ISA fija y mediante un método de multiplexación conseguir la funcionalidad del conjunto.

%nexo con neorv32

\subsection{NEORV32}

\label{neorv32}

%foto opcode cfu
    
El NEORV32 es un microcontrolador SoC personalizable \textit{Open Source} construido entorno a la CPU RISC-V homónima NEORV32.
El repositorio principal del proyecto comenzó el 23 de junio de 2020 con un primer \textit{commit} de su creador y máximo contribuidor Stephan Nolting.

Para la elaboración de este proyecto de investigación resulta de gran importancia conocer los métodos de conexión disponibles en el NEORV32.
A este respecto, los principales modos de acoplamiento de coprocesadores con los que cuenta este microcontrolador son los siguientes:

\begin{itemize}
    \item \textit{Stream Link Interface} (SLINK)
    \item \textit{Processor-External Bus Interface} (XBUS)
    \item \textit{Custom Functions Subsystem} (CFS)
    \item \textit{Custom Functions Unit} (CFU)
\end{itemize} 

Se procede a hacer un breve descripción de cada uno de ellos.
En primer lugar, \textit{Stream Link} es una interfaz para realizar transmisiones \textit{stream} compatible con un subconjunto de AXI4-Stream \cite{axi}.
Esta interfaz proporciona canales RX y TX unidireccionales e independientes para enviar y recibir un flujo de datos.
Cada canal dispone de una FIFO interna configurable para almacenar los datos provenientes del \textit{stream}.
En segundo lugar, \textit{Processor-External Bus} es una interfaz de bus general para acoplar aceleradores mapeados en memoria compatible con Wishbone \cite{wb} y AXI4-Lite \cite{axi-lite}.
Esta interfaz cuenta con un módulo de caché opcional denominado \textit{X-CACHE}.
En tercer lugar, \textit{Custom Functions Subsystem} es un \textit{template} vacío que proporciona hasta 64 registros de lectura/escritura de 32 bits mapeados en memoria a los que la CPU puede acceder mediante operaciones normales de carga/almacenamiento.
En cuarto lugar, \textit{Custom Functions Unit} es una unidad funcional que se integra directamente en el \textit{pipeline} de la CPU.
Fue añadida al NEORV32 basándose en el borrador de la especificación CFU/CXU.
Esto permitió implementar instrucciones RISC-V personalizadas. 
Los formatos de instrucción soportados por el NEORV32 son \textit{R3-Type}, \textit{R4-Type} y \textit{R5-Type} \footnote{Cabe destacar que recientemente Stephan a eliminado el soporte para la instrucción \textit{R5-Type} \href{https://github.com/stnolting/neorv32/pull/971}{\#971}}.
Los dos primeros formatos son un estándar de RISC-V y el último es exclusivo de NEORV32. 
En concreto, el primer y el segundo formato permiten direccionar dos y tres registros de entrada de 32 bits, respectivamente. 
Además, la función se seleccionada a través de \textit{funct7} y/o \textit{funct3} en el primer caso y a través de \textit{funct3} en el segundo caso. 
El tercer formato permite direccionar cuatro registros de entrada de 32 bits. 
Al no disponer del campo \textit{funct3} y/o \textit{funct7} solo se pueden realizar dos funciones personalizadas a través de este formato, \textit{A format} y \textit{B format}.  
La figura \ref{fig:ins} muestra los tipos de instrucciones personalizadas de 32 bits soportadas por NEORV32 a través de la extensión ISA específica Zxcfu.

\begin{figure}[h!]
    \centering
    \includegraphics[width=14cm]{Figuras/CFU_INS.pdf}
    \caption{Tipos de instrucciones CFU \textit{custom} para el caso del NEORV32.}
    \label{fig:ins}
\end{figure}

A lo largo de estos cuatro años el proyecto ha estado en constante evolución.
A este respecto, son destacables las contribuciones que realizó el miembro de GDED Unai Martinez Corral (\textit{umarcor}) a lo largo del 2021 y 2022 entorno al tema de la integración continua del proyecto, además de llevar acabo su sugerencia de separar del repositorio principal el repositorio \textit{neorv32-setups}.
%mas cosas k no solapen con el punto 2.1 
En el transcurso del desempeño de este proyecto de investigación, se han visto aplicar numerosos cambios.
Con respecto a los modos de conexión se destacan las siguientes actualizaciones:
añadir a la interfaz SLINK la señal AXI-stream-compatible \textit{tlast} \href{https://github.com/stnolting/neorv32/pull/815}{\#815}, renombrar la interfaz de bus externa de \textit{Processor-External Memory Interface (WISHBONE)} a \textit{Processor-External Bus Interface (XBUS)} \href{https://github.com/stnolting/neorv32/pull/846}{\#846}, reajustar la interfaz CFU \href{https://github.com/stnolting/neorv32/pull/932}{\#932}.
Estas actualizaciones ejemplifican la constante transformación del proyecto. 
En el apartado \ref{Selec} se detallan los argumentos por los que se ha escogido este microcontrolador.

\subsection{CRI}

\label{cri}

Desde finales del siglo pasado, investigadores de multitud de universidades, incluyendo la universidad del País Vasco, han indagado sobre la posibilidad de externalizar cálculos, de procesadores de propósito general a aceleradores hardware específicos.
En 1999, el profesor Koldo Basterretxea, presentó un nuevo método recursivo para la aproximación de funciones basado en el uso de operadores reticulares modificados (MLO) \cite{basterretxea1999pwl} \cite{tarela2002optimised}.
El objetivo era utilizar este método como un marco general para realizar futuras implementaciones, especialmente enfocadas a los sistemas difusos (\textit{fuzzy}), así como a las redes neuronales artificiales (RNA).
Este método, denominado Interpolación Recursiva Centrada (CRI), es un algoritmo que opera nuevas funciones afines de interpolación a cada nivel de recursión.
Para hacerlo de manera sencilla y tratar de minimizar los parámetros del algoritmo, CRI genera funciones de interpolación centradas.

\begin{equation}\label{ec:1}
h_{1} = \frac{1}{2} (y_{1} + y_{2} - \Delta)\tag{*}
\end{equation}

Atendiendo a la ecuación \eqref{ec:1} encontramos el parámetro $\Delta$, denominado profundidad de interpolación.
CRI utiliza valores de $\Delta$ optimizados en cada nivel de recursión, de manera que se minimiza el error máximo de la aproximación.
En este sentido, para la aproximación mediante CRI de funciones paramétricas (con pendiente y saturación ajustable), se demuestra que el $\Delta$ optimizado es constante cuando variamos la pendiente.
Para el caso de la función paramétrica sigmoide \eqref{ec:2}, el $\Delta$ optimizado toma los siguientes valores para las 3 primeras recursiones:

\begin{table}[h!]
\centering
\caption{Tres primeras recursiones del $\Delta$opt (función sigmoide).}
\label{tab:1}
\begin{tabular}{|c|c|}
\hline
Sigmoide & $\Delta$opt    \\ \hline
q=1      & 0.3091 \\ \hline
q=2      & 0.2811 \\ \hline
q=3      & 0.2654 \\ \hline
\end{tabular}
\end{table}

\begin{equation}\label{ec:2}
\sigma(x)= \frac{1}{1 + e^{-x}}\tag{**}
\end{equation}

Desde un punto de vista hardware, Basterretxea ha realizao descripciones de diseños empleando este algoritmo desde el inicio de su planteamiento teórico.
A este respecto, en el repositorio del grupo de investigación, hay disponible un acelerador configurable y programable de funciones de activación basado en CRI.
Dicho acelerador soporta 7 tipos de FAs: sigmoide, tangente, lineal, ReLu, Hardlimit, satlin y leakyReLu. 
Para el presente trabajo de investigación, se ha utilizado exclusivamente la lógica referente a la función sigmoide.

\section{Anteproyecto}

\label{antepro}

%practicas no remuneradas
Cabe destacar que este proyecto se ha desempeñado en el marco de unas prácticas remuneradas tituladas \textit{Desarrollo de Custom Functional Units de IA para RISC-V}.
La duración de estas prácticas ha sido de 5 meses (594 horas) y por realizarlas se ha recibido una bolsa de ayuda total de 2000€.
No obstante, previas a estas prácticas remuneradas, se realizaron otras prácticas obligatorias relativas al \textit{practicum} del Máster.
Estas también se llevaron a cabo en el grupo de investigación GDED bajo el título \textit{Diseño e integración de CFU para RISC-V en FPGA}.
Tuvieron una duración de 2 meses y 20 días (225 horas) y fueron calificadas con una matrícula de honor (10).
Las tareas a desarrollar fueron las siguientes:

\begin{itemize}
    \item Realizar un \textit{set-up} de herramientas FOSS (\textit{Free and Open-Source Software}) para la implementación y simulación del proyecto.
%meter FOSS en acrónimos
    \item Implementar el \textit{soft-core} NEORV32 en FPGA (Arty A7 35t y 100t) y verificarlo.
    \item Experimentar con distintas opciones de integración de CFUs para NEORV32 y evaluar.
    \item Generar documentación. 
\end{itemize} 

El desarrollo de estas labores sirvió para tomar contacto con el ecosistema RISC-V. 
En concreto, con el micropocesador NEORV32.
A este respecto, se comenzó a experimentar con los diferentes modos de acoplamiento de coprocesadores con los que cuenta este micro.
Además, se realizaron los primeros ensayos de integración de este \textit{soft-core} en FPGA.
Con relación a esto, se comenzó a investigar sobre herramientas libres (FOSS) para realizar la implementación en placa de los diseños hardware.
%retocar-listo
Esto fue motivado porque hasta entonces únicamente se utilizaban las herramienta privativas que ofrece Xilinx.
Asimismo, se documentó los avances relativos al proyecto en forma de \textit{issues} en el repositorio de GitLab asociado a este proyecto de investigación.
Más tarde se trasladó parte de esta información a la pagina web vinculada a dicho repositorio.

En definitiva, las tareas desarrolladas en estas prácticas obligatorias se consideran el anteproyecto de esta investigación.

\section{Objetivos y alcance del proyecto}

%paper dcis tal noseke-mejor en beneficios

El objetivo principal de este proyecto es integrar unidades coprocesadoras para IA en un núcleo RISC-V, en concreto en el \textit{soft-core} NEORV32. 
En este contexto, se realizará la caracterización del rendimiento de cada uno de los modos de conexión con objeto de determinar la forma óptima de integración.
Además, se deberá verificar su operatividad mediante la implementación en FPGA.

Los objetivos secundarios son:
\begin{itemize}
    \item Realizar aportaciones al estado del arte referente al \textit{soft-core} NEORV32 en forma de contribuciones a su repositorio oficial de GitHub.
    \item Realizar aportaciones al estado de la investigación sobre el uso de procesadores RISC-V realizada por GDED.
Específicamente, mediante contribuciones de código, \textit{pipelines} de integración continua y documentación en el repositorio del grupo relativo al NEORV32. 
Pretendiendo que este hecho facilite en un futuro la investigación de  otros alumnos/as.
\end{itemize} 

Respecto al alcance del proyecto, se definen con precisión los elementos incluidos dentro de este proyecto de investigación: 

\begin{itemize}
    \item La carectirazación mediante simulación, en términos de rendimiento y throughput, de los siguientes métodos de acoplamineto disponibles en el NEORV32:
        \begin{itemize}
            \item \textit{Stream Link Interface}
            \item \textit{Processor-External Bus Interface} 
            \item \textit{Custom Functions Subsystem}
            \item \textit{Custom Functions Unit}
        \end{itemize} 
    \item La integración mediante \textit{Custom Functions Unit} de un coprocesador hardware embebido para aplicaciones de IA. 
        \begin{itemize}
            \item El coprocesador integrado únicamente acelera los cálculos de la activación a través de la función sigmoide.
        \end{itemize} 
    \item La comparación mediante simulación de los ciclos de reloj necesarios para calcular la activación mediante la integración micro más acelerador versus micro haciendo uso exclusivamente de sus funcionalidades por defecto.
    \item La implementación en FPGA de todos los ensayos realizados en simulación para la verificación de su correcta operatividad. 
\end{itemize} 

Queda excluido del alcance del proyecto la integración del resto de funciones de activación que ofrece el acelerador configurable y programable basado en CRI.
También queda excluida la última etapa definida en el informe de viabilidad de la propuesta de TFM que pretendía comparar en términos de rendimiento un modelo completo de RNA computado por software con el mismo modelo computado en un NEORV32 incorporando las instrucciones de aceleración del cálculo de las FAs.

\section{Beneficios que aporta el trabajo}

\label{ben}

% beneficios inmediatos nombrar las pull request realizadas al repo de neorv32

%Captura contribuidor

%a largo plazo aportar al estado del arte de los modelos de calculo ai at the edge

%Bajo los objetivo desarrollo sostenible de tal y cual pork...

Se consideran beneficios aportados por el presente proyecto de investigación las siguientes contribuciones al repositorio principal del NEORV32:

\begin{itemize}
    \item \textit{Pull request} \href{https://github.com/stnolting/neorv32/pull/717}{\#717}: \textit{Fix bug in neorv32\_slink\_available() function}
    \item \textit{Pull request} \href{https://github.com/stnolting/neorv32/pull/722}{\#722}: \textit{Fix-up the litex wrapper}
    \item \textit{Pull request} \href{https://github.com/stnolting/neorv32/pull/727}{\#727}: \textit{Fix comment mistake}
    \item \textit{Pull request} \href{https://github.com/stnolting/neorv32/pull/891}{\#891}: \textit{Fix UART receiver}
\end{itemize} 

En relación a estas aportaciones, el autor de este TFM (bajo el alias \textit{Unike267}) se encuentra como el contribuidor número 16 de 37 del repositorio principal del NEORV32, a fecha 11 de septiembre de 2024. Este hecho se refleja en la figura \ref{fig:con}.

\begin{figure}[h!]
    \centering
    \includegraphics[width=14cm]{Figuras/Contributor.png}
    \caption{\textit{Unike267} como contribuidor del repositorio principal del NEORV32.}
    \label{fig:con}
\end{figure}

Además, cabe destacar que los resultados obtenidos mediante la caracterización de los diferentes métodos de acoplamiento de coprocesadores con los que cuenta el NEORV32 se han plasmado en un \textbf{artículo de congreso}. 
Este artículo se ha presentado en el marco del congreso DCIS 2024.
La información relativa a este congreso, así como el propio artículo, se proporciona en el apéndice \ref{Articulo}.
Puesto que dicho artículo ha sido aceptado, se publicará en el IEEEXplore.
Se considera un beneficio aportado por este trabajo la contribución en dicha base de datos.  

También, se considera un beneficio aportado por este trabajo la publicación del contenedor \cite{gh:conatiner-implarty} desarrollado para realizar la síntesis, implementación y generación de \textit{bitstream} para las FPGAs Arty A7 35T y 100T mediante las herramientas \textit{Open Source}: GHDL + yosys + GHDL yosys plugin + nextpnr-xilinx + prjxray.

En lo que respecta a los Objetivos de Desarrollo Sostenible (ODS), este trabajo de investigación pretende aportar beneficios en el marco del objetivo \textit{Industria, innovación e infraestructura} (ODS 9) y del objetivo \textit{Reducción de las desigualdades} (ODS 10). 
Respecto al ODS 9, se considera que la investigación entorno al enfoque distribuido para realizar IA en el borde supone fomentar la innovación en lo que respecta a este ámbito.
Respecto al ODS 10, se considera que distribuir contenedores para facilitar el uso de herramientas \textit{Open Source}, así como contribuir a proyectos de hardware libre que puedan derivar a la fabricación de chips \textit{royalty-free}, fomenta la reducción de las desigualdades.

\section{Análisis del estado del arte}

\label{sta-art}

Haciendo un repaso por las principales bases de datos de ámbito electrónico, se observan varios proyectos con una filosofía similar a la presentada en este trabajo de investigación.
Un ejemplo muy interesante es el encontrado en una publicación titulada \textit{Tiny Neuron Network System based on RISC-V Processor: A Decentralized Approach for IoT Applications} \cite{9942990}.
En dicho artículo, se presenta una investigación sobre un pequeño acelerador de redes neuronales en un SoC basado en RISC-V para acelerar una IA empleada en aplicaciones de IoT. 
Este coprocesador implementa una MAC (multiplicador y acumulador) de precisión variable en bits o una MAC estocástica para reducir el área de hardware y el consumo de energía. 
Es curioso el hecho de que se emplea la misma tecnología FPGA que en el presente proyecto, una Arty A7 100T.
Los resultados presentados en este trabajo son destacables, consiguiendo realizar con precisión de 8 bits redes neuronales convolucionales (CNN) con una precisión del 98,55\%.
En su caso, el método optado para acoplar los aceleradores ha sido una interfaz AXI.

Otro ejemplo muy interesante es el expuesto en una publicación titulada \textit{CNN Specific ISA Extensions Based on RISC-V Processors} \cite{9802445}.
En ella, se presenta una extensión de instrucciones basada en la ISA RISC-V destinada a aumentar la eficiencia computacional de las CNN en dispositivos en el borde. 
En su caso emplean el core RISC-V \textit{Open Source} Zero-riscy \cite{8106976} \cite{gh:zero-riscy}.
Con objeto de evaluar el efecto de la extensión de instrucciones propuesta, se realiza una serie de cargas de trabajo en el núcleo de referencia y en el ampliado. 
Se obtinen un ratio de aceleración de 1,5× cuando se ejecuta una CNN, y alcanza 2,48×-2,82× cuando sólo se realizan los cálculos de convolución. 
Los resultados obtenidos en este artículo demuestran que las extensiones de ISA propuestas pueden mejorar eficazmente el rendimiento de las CNN.

Los casos mencionados, así como otros de gran interés, se encuentran recogidos en una publicación titulada \textit{A Review of Edge Intelligence Applications Based on RISC-V} \cite{10336594}.
En ella se resume el uso de RISC-V en aplicaciones de IA en el borde desde un punto de vista hardware y software. 
Además, a lo largo de la revisión se analizan varios aceleradores, coprocesadores, compiladores y \textit{toolchains} basados en RISC-V, así como las principales aplicaciones software de la IA en el borde. 

A lo largo del transcurso de este proyecto se ha preferenciado, en la medida de lo posible, el uso de herramientas libres.
En este sentido, las herramientas EDA FLOS y las propietarias/comerciales no son ecosistemas aislados.
Al contrario, en los últimos años se han visto colaboraciones de proyectos \textit{Open Source} con iniciativas privativas.
Un ejemplo de ello es la integración de la herramienta RapidWright \cite{gh:rapid} a la \textit{Suite} de diseño Vivado.
En concreto, este proyeto \textit{Open Source} desarrollado por \textit{AMD Research and Advanced Development} tiene como objetivo permitir a los usuarios avanzados una mayor flexibilidad a la hora de personalizar sus soluciones mediante una metodología de diseño utilizando módulos pre-implementados.
Además, se han realizado concursos \cite{contest} patrocinados por AMD con el objetivo de promover y demostrar que el Formato de Intercambio de FPGA (FPGAIF - \textit{FPGA Interchange Format}) \cite{FPGAIF} es una representación intermediaria eficiente y robusta para trabajar en problemas de \textit{backends} de FPGAs, incluso a escala industrial.
Asimismo, este tipo de iniciativas también tratan de fomentar la innovación de algoritmos de enrutamiento de FPGAs que den prioridad al tiempo de ejecución, con objeto de posibilitar su aplicación en la emulación de ASICs.
Cabe destacar que el FPGAIF es un estándar de formato de intercambio diseñado para proporcionar toda la información necesaria mediante la cual realizar el \textit{place and route} en un contexto \textit{Open Source}.
En la misma línea, Siemens ha observado un crecimiento saludable entorno a la Metodología de Verificación VHDL de Código Abierto (OSVVM - \textit{Open Source VHDL Verification Methodology}) \cite{osvvm} y la Metodología de Verificación Universal VHDL (UVVM - \textit{Universal VHDL Verification Methodology}) \cite{uvvm} desde 2018, lo que en sus propias palabras \say{es alentador} \cite{wilson-research}.
Por lo tanto, a la vista de estos ejemplos, podemos afirmar que los comerciales tradicionales de herramientas EDA están empezando a facilitar el uso de herramientas FLOS e incluso a integrar parte o la totalidad de las mismas en sus propuestas comerciales.
Este hecho refleja un futuro híbrido en lo referente al ecosistema de herramientas para FPGAs.

\section{Análisis de alternativas}

Alternativamente, la selección del microcontrolador se puede afrontar desde dos puntos de vista.
El primero de ellos es seguir defendiendo la elección desde una perspectiva \textit{Open Source}.
En este sentido, se puede elegir otra de las opciones RISC-V propuestas en la sección \ref{eco-risc}.
Sin embargo, también se puede optar por un microcontrolador basado en otra ISA libre, como alguno de las nombrados en la sección \ref{isas-free}.
El segundo punto de vista es claudicar con la iniciativa de emplear un proyecto \textit{royalty-free}.
Es decir, trasladar la búsqueda a un \textit{soft-core} privativo sujeto al modelo de propiedad intelectual.
A este respecto, se puede optar por MicroBlaze™ \cite{microblaze}, ya que la entidad desarrolladora de su diseño es Xilinx (ahora AMD), se ve adecuado utilizar un \textit{soft-core} realizado por la misma empresa encargada de fabricar las FPGAs utilizadas en este proyecto de investigación.
Como curiosidad, cabe mencionar que debido a la enorme importancia que ha supuesto la entrada de RISC-V al contexto de ISAs actuales, AMD ofrece un \textit{soft-core} RISC-V llamado MicroBlaze™ V \cite{microblazeV}.
Esta tesitura sirve para ejemplificar que un proyecto por el hecho de ser \textit{Open Source}, no impide que se desarrollen iniciativas comerciales que lo exploten.
No obstante, es labor de las personas que defendemos la filosofía del software/hardware libre aportar opciones que supongan una competencia real frente a estas iniciativas propietarias.
Además, este punto de vista privativo supondría un cambio de paradigma que incurriría en dejar de fomentar el ODS 10, defendido en el apartado \ref{ben}.

Asimismo, existe la alternativa de emplear el \textit{framework} LiteX \cite{gh:litex} para llevar a cabo la implementación del NEORV32.
Entre otras cosas, LiteX ofrece un conjunto de periféricos, como controladores DRAM, PCI, Ethernet, SATA etcétera, que junto a la descripción hardware del propio microcontrolador permiten multiplicar sus posibilidades de aplicación.
Cabe destacar que la lógica digital de los complementos que ofrece LiteX están descritos en Migen \cite{gh:migen}.
Este hecho no impide integrar en él código en otros HDLs.
De igual modo, es común generar diseños de LiteX en HDLs más tradicionales, como verilog.

Por último, se encuentra la alternativa de comenzar la aproximación de IA a en el borde mediante la externalización a hardware específico de otro cálculo relativo a las RNAs.
Esta alternativa es bastante diversa y podría ir desde acelerar la activación mediante otra función disponible en el coprocesador configurable y programable basado en CRI hasta elegir otra operación para ser acelerada.
Un ejemplo de esta última propuesta es la detallada en una publicación titulada \textit{A Soft RISC-V Vector Processor for Edge-AI} \cite{9885953}.
En ella se presenta una unidad vectorial basada en un \textit{array} sistólico que está estrechamente integrada en el pipeline de un núcleo RISC-V de 32 bits. 
Tras evaluar el rendimiento de este enfoque distribuido en una FPGA Xilinx Virtex 7, se concluye un aumento de velocidad de hasta 40,7 veces con respecto al núcleo escalar RISC-V en tareas de reconocimiento de imágenes, a costa de un aumento en el consumo energético y en los recursos hardaware de 1,2 y 1,8 veces respectivamente. 

\section{Descripción de la solución propuesta}

%Solucion propuesta para realizar IA en el borde es externalizar tal. FA sigmoide. El desarrollo descrito en el cap 2.

La principal tesis que defiende este TFM es externalizar la computación referentes a las RNA, del procesador a hardware específico.
Este hecho tiene como objetivo distribuir la gestión computacional en pos de implementar una primera aproximación de IA en el borde.
En este sentido, se propone utilizar un coprocesador basado en el método CRI y acoplarlo a un procesador RISC-V.
Mediante los argumentos descritos en la sección \ref{Selec} se elige el microcontrolador NEORV32. 
Con objeto de generar un criterio de selección del modo de acoplamiento, se propone realizar una caracterización del rendimiento de los principales métodos de conexión con los que cuenta este microcontrolador.
Esta caracterización se detalla en la sección \ref{Carac}.
Por último, se acopla el coprocesador encargado de acelerar el cálculo de la FA sigmoide basado en CRI al NEORV32. 
Asimismo, se verifica el beneficio de emplear este tipo de enfoque distribuido comparándolo con realizar los mismos cálculos utilizando únicamente los recursos predeterminados del microcontrolador.
Esta integración se detalla en la sección \ref{Integ}.
Se procede a concretar los pasos seguidos a lo largo del desarrollo \ref{Desarrollo}.

El primer paso para llevar a cabo este proyecto es realizar la selección del microcontrolador.
Se requiere de su implementación en FPGA, por lo que se necesita la descripción en HDL de su arquitectura.
Es interesante destacar que el término empleado para referirse a este formato es \textit{soft-core} (núcleo blando).
A diferencia de los micros \textit{hard-core} (núcleo duro), es decir, los impresos en silicio, los \textit{soft-cores} están pensados para ser implementados en dispositivos lógicos programables. 
Además, no solo se precisa de la descripción hardware del micro sino también de las herramientas asociadas para compilar software interpretable por el mismo.
En este sentido, la búsqueda se ha centrado en un proyecto \textit{royalty-free}.

El siguiente paso es evaluar los diferentes modos de acoplamiento de coprocesadores con los que cuenta el micro seleccionado.
Con objeto de caracterizar el rendimiento de cada uno de ellos y concretar el más adecuado para nuestro propósito.
Para ello, se testean varios aceleradores acoplados al micro mediante todos los modos de los que este dispone.
De esta manera, se obtiene un conjunto de bancos de prueba para ensayar en simulación y así proporcionar las lecturas de latencia y throughput de cada uno de los métodos de conexión.
En concreto, se han acoplado 3 multiplicadores con diferentes características.
Además, todos los ensayos se han implementados en FPGA para verificar su correcta operatividad.

El último paso es comenzar a abordar un enfoque distribuido para realizar IA en el borde. 
A este respecto, se requiere de aceleradores embebidos que realicen los cálculos asociados a las redes neuronales.
Atendiendo a la caracterización realizada en el paso anterior, se establece un criterio mediante el cual decidir que modo de conexión emplear para acoplar estos diseños embebidos.
Una vez acoplados, se corrobora el beneficio de utilizar coprocesadores frente a realizar los cálculos íntegramente mediante las funcionalidades predeterminadas del micro.
Para ello, se realiza un banco de pruebas que compare en simulación y en implementación los ciclos de ejecución de ambos planteamientos.
Dado el limitado tiempo de investigación tan solo se ha externalizado el cálculo de la activación a través de la función sigmoide.
Para lo cual, se ha utilizado un acelerador embebido CRI diseñado por Koldo Basterretxea.


\chapter{Desarrollo} % Título del capítulo

\label{Desarrollo}

\section{Selección del microcontrolador}

\label{Selec}

Resulta imprescindible que el microcontrolador seleccionado para este proyecto de investigación cumpla con los siguientes requisitos:

%cambiar-listo!
\begin{itemize}
    \item Estar basado en una ISA RISC-V, ya que el contexto del proyecto está orientado a acoplar coprocesadores en núcleos RISC-V.
    \item Estar descrito en el lenguaje de descripción de hardware VHDL, ya que los coprocesadores embebidos para aplicaciones de IA que se pretenden acoplar están descritos en este lenguaje.
    \item Contar con extensión de instrucciones para conectar coprocesadores mediante CFU, además de con soporte para comunicaciones \textit{memory-mapped} e interfaces \textit{stream}, ya que se necesita una variedad de métodos de conexión para realizar la caracterización del rendimiento mediante diferentes modos.
\end{itemize} 

En este sentido, el NEORV32 \cite{gh:neorv32} cumple con todos estos requerimientos. 
Además, en la plataforma de desarrollo colaborativo donde está alojado, cuenta con una comunidad muy activa.
Es por ello que se encuentra bajo una revisión constante de \textit{bugs} (fallos), tanto por parte del autor como de los usuarios.
De esta manera, se asegura en gran medida la correcta operatividad del mismo.
Además, el autor se dedica a realizar actualizaciones periódicas de sus funcionalidades.
Por si fuera poco, tanto el autor como la comunidad tienen una gran disponibilidad para responder dudas sobre temas relacionados con el proyecto, lo que resulta de gran ayuda.
Con respecto a la compilación de lenguajes de alto nivel, el proyecto ofrece \textit{toolchains} precompiladas de RISC-V para GCC.  
Estas herramientas permiten hacer compilación cruzada de C/C++ a instrucciones de RISC-V  en un entorno Linux \cite{gh:neorv32-tool}.
Cabe destacar que también se facilita un contenedor para realizar esta tarea \cite{gh:sim-conatiner}. 
Además, cuenta con un soporte de librerías para compilar funciones software específicas de NEORV32. 
Asimismo, el repositorio ofrece una variedad de ejemplos de aplicación software de todos los recursos con los que cuenta el micro.
En adición a todo lo mencionado, este microcontrolador cuenta con un \textit{datasheet} \cite{neorv32-ds} y una \textit{user guide} \cite{neorv32-ug} realizadas por el autor y actualizadas a la par que el código del proyecto, las cuales destacan por su calidad.
Teniendo en cuenta todas estas consideraciones, el NEORV32 es el procesador seleccionado para este proyecto.

\hspace{10 mm}

\section{\textit{Workflow}}

\label{Workf}

Las herramientas EDA FLOS y las propietarias/comerciales no son ecosistemas aislados.
Al contrario, en los últimos años se han visto colaboraciones de proyectos \textit{Open Source} con iniciativas privativas.
En este sentido, se puede destacar la integración de la herramienta RapidWright \cite{gh:rapid} a la \textit{Suite} de diseño Vivado.
En concreto, este proyeto \textit{Open Source} desarrollado por \textit{AMD Research and Advanced Development} tiene como objetivo permitir a los usuarios avanzados una mayor flexibilidad a la hora de personalizar sus soluciones mediante una metodología de diseño utilizando módulos pre-implementados.
En adición a esto, se han realizado concursos \cite{contest} patrocinados por AMD con el objetivo de promover y demostrar que el \textit{FPGA Interchange Format} (FPGAIF - Formato de Intercambio de FPGA) \cite{FPGAIF} es una representación intermediaria eficiente y robusta para trabajar en problemas de \textit{backends} de FPGAs, incluso a escala industrial.
Además, este tipo de iniciativas también tratan de fomentar la innovación de algoritmos de enrutamiento de FPGAs que den prioridad al tiempo de ejecución, con objeto de posibilitar su aplicación en la emulación de ASICs.
Cabe destacar que el FPGAIF es un estándar de formato de intercambio diseñado para proporcionar toda la información necesaria mediante la cual realizar el \textit{place and route} en un contexto \textit{Open Source}.
En la misma línea, Siemens ha observado un crecimiento saludable entorno a la \textit{Open Source VHDL Verification Methodology} (OSVVM - Metodología de Verificación VHDL de Código Abierto) \cite{osvvm} y la \textit{Universal VHDL Verification Methodology} (UVVM -  Metodología de Verificación Universal VHDL) \cite{uvvm} desde 2018, lo que en sus propias palabras \say{es alentador} \cite{wilson-research}.
Por lo tanto, a la vista de estos ejemplos, podemos afirmar que los comerciales tradicionales de herramientas EDA están empezando a facilitar el uso de herramientas FLOS e incluso a integrar parte o la totalidad de las mismas en sus propuestas comerciales.
Este hecho refleja un futuro híbrido en lo referente al ecosistema de herramientas para FPGAs.

\begin{figure}[h!]
    \centering
    \includegraphics[width=14cm]{Figuras/workflow.pdf}
    \caption{\textit{Workflow} del \textit{Setup} personalizado.}
    \label{fig:workf}
\end{figure}

Atendiendo a este nuevo paradigma híbrido de herramientas, en el presente trabajo de investigación se propone el uso tanto de herramientas FLOS como privativas.
Como se observa en la figura \ref{fig:workf}, el flujo de trabajo propuesto es el siguiente:

\begin{itemize}
    \item Compilación: respecto a la compilación de software en C que posteriormente se cargará en la memoria de instrucciones del NEORV32, se emplea exclusivamente la herramienta FLOS GCC.
    \item Implementación: respecto a las implementaciones en placa de los diseños, se efectúan todas ellas tanto para la Arty A7 35T como para la Arty A7 100T y mediante dos vías paralelas:
        \begin{itemize}
            \item Haciendo uso del conjunto de herramientas FLOS: GHDL \cite{gh:ghdl}, yosys \cite{gh:yosys}, GHDL yosys plugin \cite{gh:ghdl-plugin}, nextpnr-xilinx \cite{gh:nextpnr} y prjxray \cite{gh:prjxray} para realizar la elaboración, la síntesis y el \textit{place and route} y de la herramienta openFPGALoader \cite{gh:openFPGALoader} para cargar el \textit{bitstream} en la placa.
            \item Haciendo uso de la \textit{Suite} de diseño privativa Vivado.
        \end{itemize}
    \item Simulación: respecto a las simulaciones realizadas a lo largo de las secciones \ref{Carac} y \ref{Integ} se emplea principalmente el \textit{framework} FLOS VUnit \cite{gh:vunit}, con el cual se realizan todas ellas.
No obstante, también se utiliza Vivado en ciertas ocasiones.  
En concreto, en los ensayos en los que se hace uso de la funcionalidad ILA.
Esta funcionalidad se ha utilizado, por ejemplo, para testear la correcta operatividad de un \textit{wrapper} Wishbone.
\end{itemize} 

El conjunto de herramientas descritas en la explicación de este flujo de trabajo no solo se utilizan a nivel local, también se utilizan, todas o parte de ellas, en \textbf{integración continua (CI)} en repositorios \textit{online}, tanto en el GitLab del grupo de investigación como en el GitHub propio.
Para ello, se utilizan varios \textbf{contenedores}.
Para la generación de \textit{bitstream} mediante herramientas FLOS, se utiliza el contenedor mencionado en la sección \ref{ben}, el cual es generado a su vez en CI.
Este contenedor se utiliza en la integración continua tanto del repositorio de GitLab como de GitHub.
Para la generación de \textit{bitstream} mediante Vivado, se utiliza una contenedor que solamente es accesible por los ordenadores del laboratorio del grupo de investigación, el cual está alojado en nuestro servidor Orion.
Esto es debido a que al ser un programa privativo no se pueden distribuir publicamente contenedores con este software. A consecuencia de ello, la generación de \textit{bitstream} mediante esta vía solo está disponible en la integración continua del repositorio del grupo (GitLab).
Para realizar los ensayos en simulación, se utilizan principalmente dos contenedores de VUnit.
El \href{https://console.cloud.google.com/gcr/images/hdl-containers/global/sim/osvb}{gcr.io/hdl-containers/sim/osvb:latest} con GHDL compilado con llvm como \textit{backend} y el \href{https://hub.docker.com/layers/ghdl/vunit/mcode-master/images/sha256-e32029c5be70a5fa0fc94bffd15d72fa8b84ad8aaf2dc7cfa8ab8324ef733ed0?context=explore}{docker.io/ghdl/vunit:mcode-master} con GHDL compilado con mcode como \textit{backend}.
Esto se debe a que la funcionalidad \href{https://github.com/stnolting/neorv32/discussions/886}{\textit{external names}}, para capturar señales de jerarquías inferiores, solo es soportada en GHDL si este está compilado con el \textit{backend} mcode.
En definitiva, haciendo uso de estos recursos mediante la metodología de integración continua, se consigue automatizar todas las simulaciones, visualizando y gestionando sus resultados, así como la generación de todos los \textit{bitstreams}, cada vez que se hace un \textit{push} al repositorio.
Cabe destacar que la compilación de software solo se realiza en local, aunque también se utiliza un contenedor \cite{gh:sim-conatiner}, no está automatizada en integración continua.

\subsection{Cargar software en el NEORV32}

Antes de entrar en los detalles del acoplamiento de periféricos \textit{custom}, se procede a realizar un repaso de como cargar un software en C al \textit{softcore} NEORV32.
Como se ha mencionado, el proyecto NEORV32 proporciona herramientas para realizar la compilación cruzada desde Linux a la arquitectura RISC-V.
Estas herramientas están acompañadas de archivos \textit{Makefiles} mediante los cuales se permiten añadir argumentos al comando \textit{Make}, con objeto de, entre otras cosas, proporcionar el programa compilado en diferentes formatos de salida.
A lo largo de esta sección, nos centraremos en tres de estos formatos:

\begin{itemize}
    \item Ejecutable, \textit{exe} (.bin)
    \item app\_image (.vhd)
    \item Hexadecimal (.hex)
\end{itemize} 

Cada una de estas salidas tiene la misma información, el programa compilado.
No obstante, cada una de ellas puede utilizarse para cargar el software en la IMEM (memoria de instrucciones) en diferentes puntos del \textit{Workflow}:

\begin{itemize}
    \item El \textit{exe} se puede cargar en el NEORV32 una vez que esté ejecutándose en la FPGA. Esta transferencia se realiza a través del \textit{bootloader}.
    \item La app\_image remplaza el contenido por defecto de una de las fuentes RTL del diseño del NEORV32, de modo que su contenido se codifica cuando este se sintetiza.
    \item El archivo .hex se lee durante la síntesis, por lo que es equivalente a la solución de la app\_image, pero no requiere modificar las fuentes RTL cada vez que se actualiza el software a cargar.
\end{itemize} 

Estas opciones se resumen en la tabla \ref{tab:2}.

\begin{table}[h!]
\centering
\caption{Tres formas de introducir software en la IMEM.}
\label{tab:2}
\begin{tabular}{|c|c|c|c|}
\hline
\textbf{Formato}  & \textbf{Comando}   & \textbf{Descripción}                                                                                            & \textbf{Bootloader}  \\ \hline
.bin              & make exe           & \begin{tabular}[c]{@{}c@{}}Después de la implementación,\\  cargar el exe mediante la CMD\end{tabular}          & Habilitado           \\ \hline
.vhd              & make image         & \begin{tabular}[c]{@{}c@{}}Antes de la síntesis, \\ sustituir la app\_image por defecto\end{tabular}            & Deshabilitado        \\ \hline
.hex              & make hex           & Durante la síntesis, leer del .hex                                                                              & Deshabilitado        \\ \hline
\end{tabular}
\end{table}

\subsubsection{\textit{Bootloader}}
\label{boot}

El NEORV32 viene por defecto con un \textit{bootloader} que se encarga de establecer la comunicación serie vía UART y generar una CMD visible desde terminales como CuteCom \cite{gh:cutecom}, \href{https://man.openbsd.org/cu.1}{cu}, o \href{https://www.gnu.org/software/screen/}{screen} en GNU/Linux.
En este sentido, hay tres formas posibles de proceder:

\begin{itemize}
    \item Deshabilitar el \textit{bootloader} y cargar/iniciar un programa desde la app\_image o desde un archivo hexadecimal.
        \begin{itemize}
            \item No se utiliza el \textit{bootloader}.
        \end{itemize} 
    \item Habilitar el \textit{bootloader} y cargar/iniciar un programa a través del \textit{Autoboot}.
        \begin{itemize}
            \item Después del \textit{reset}, cuando el \textit{bootloader} está habilitado, la primera secuencia que ocurre es el \textit{Autoboot}. 
Esta secuencia intenta obtener una imagen de arranque válida desde la flash SPI externa.
Si se encuentra una imagen válida que se pueda transferir correctamente a la IMEM (memoria de instrucciones), se inicia automáticamente la aplicación.
No obstante, si han pasado 8 segundos y no se ha detectado ninguna flash SPI o no se encuentra ninguna imagen de arranque válida, se mostrará el código de error \say{ERR EXE}, bloqueando la ejecución.
Sin embargo, durante esos 8 segundos, se puede detener la secuencia del \textit{Autoboot} pulsando cualquier tecla. 
De esta manera, se pone a disposición una CMD lista para recibir comandos.
        \end{itemize} 
    \item Habilitar el \textit{bootloader} y cargar/iniciar un programa a través de comandos en la CMD.
Los comandos soportados son los siguientes:
        \begin{itemize}
            \item \say{h} - Muestra el texto de ayuda.
            \item \say{r} - Reiniciar el \textit{bootloader}.
            \item \say{u} - Cargar un programa en formato ejecutable (\textit{neorv32\_exe.bin}) a la IMEM.
            \item \say{s} - Almacenar un ejecutable en flash SPI.
            \item \say{l} - Cargar un ejecutable desde flash SPI.
            \item \say{x} - Arrancar un programa desde flash a través de XIP.
            \item \say{e} - Iniciar un programa almacenado en la IMEM.
        \end{itemize} 
\end{itemize} 

Para elegir una de estas tres formas de proceder, se debe entender que el \textit{bootloader} es útil/necesario cuando:

    \begin{itemize}
        \item La FPGA utilizada no permite inicializar la memoria en el \textit{bitstream}. 
En consecuencia, no es posible cargar/arrancar programas a través de la app\_image.
Este es el caso de las FPGAs con SPRAM, como la Lattice ICE40 (UP3K, UP5K).
        \item Múltiples programas deben ser cargados/arrancados durante el desarrollo, sin resintetizar el diseño.
    \end{itemize} 

En la figura \ref{fig:boot} se muestra como cargar/iniciar un programa ejecutable (.exe) al NEORV32 mediante la CMD proporcionada por el \textit{bootloader}. 
Concretamente, se utiliza la terminal CuteCom \footnote {En CuteCom, el archivo que se carga a la terminal debe ser de tipo \textit{Plain} (como se muestra en la figura \ref{fig:boot}), de lo contrario se dará el error \say{ERR EXE}.}, en ella se emplean sucesivamente los comandos \say{u} (\textit{upload} - cargar) y \say{e} (\textit{execute} - ejecutar).

\begin{figure}[h!]
    \centering
    \includegraphics[width=14cm]{Figuras/cutecom_cmd_upload.png}
    \caption{Cargar un \textit{exe} a través del \textit{bootloader} de NEORV32 (terminal CuteCom).}
    \label{fig:boot}
\end{figure}

\subsubsection{Habilitar/Deshabilitar el \textit{Bootloader}}

Si el \textit{bootloader} no es útil/necesario para nuestra aplicación tendremos que considerar lo siguiente.
La IMEM se puede implementar de dos formas, como una RAM vacía o como una ROM inicializada a través del archivo que contiene el programa compilado, ya sea la \textit{neorv32\_application\_image.vhd} o el hexadecimal. 
Con el genérico \mintinline[breaklines]{vhdl}{IMEM_AS_IROM} se selecciona la implementación de la IMEM mediante una de estas dos opciones. 
Este genérico es
 
\hspace{35mm} \mintinline[breaklines]{vhdl}{IMEM_AS_IROM => imem_as_rom_c} 

\noindent y se define como 

\hspace{17mm} \mintinline[breaklines]{vhdl}{imem_as_rom_c : boolean := not INT_BOOTLOADER_EN;}

\noindent Por lo tanto, para cargar un programa desde la \textit{neorv32\_application\_image.vhd} (o desde el hexadecimal), la IMEM debe implementarse como una ROM inicializada mediante ese archivo, por lo que el \textit{bootloader} \textbf{debe estar deshabilitado}.
Se discutió con Stephan \href{https://github.com/stnolting/neorv32/discussions/824}{(\#824)} acerca de por qué la IMEM se inicializa como una RAM vacía cuando el \textit{bootloader} está activado. 
Y según el diseñador del NEORV32, \say{si la IMEM se implementara como una RAM preinicializada, entonces la imagen podría corromperse durante el tiempo de ejecución (imagina algún puntero deshonesto escribiendo en la IMEM), lo que requeriría volver a cargar el programa original. 
Por lo tanto, la carga del \textit{bootloader} se requeriría de todos modos.}

El proceso para deshabilitar el \textit{bootloader} es sencillo, en el TOP del diseño del NEORV32, se debe cambiar la constante \mintinline[breaklines]{vhdl}{INT_BOOTLOADER_EN} de \textit{true} a \textit{false}, como se muestra en el extracto de código \ref{code:1}.

\begin{code}
\begin{minted}[frame=lines,framesep=2mm,baselinestretch=1.2,fontsize=\footnotesize]{vhdl}
neorv32_top_inst : neorv32_top
generic map(
----------------------------------
INT_BOOTLOADER_EN         => false,
----------------------------------
)
\end{minted}
\caption{Constante para deshabilitar el \textit{bootloader}.}
\label{code:1}
\end{code}

\subsubsection{Cargar un programa compilado desde un archivo hexadecimal}

Como se ha mencionado, en vez de cargar un programa compilado desde el archivo \textit{neorv32\_application\_image.vhd}, es posible cargar el programa compilado desde un archivo hexadecimal (.hex).
Para ello, se necesitan hacer unas pequeñas modificaciones en el código HDL del NEORV32.
En particular, se debe añadir una nueva función en el paquete \textit{neorve32\_package.vhd}.
Esta función se encargará de leer el archivo hexadecimal usando la librería \textit{std.textio.all}. \footnote{Esta librería está soportada desde la versión VHDL 2008.}
La función en cuestión es la descrita en el extracto de código \ref{code:2}.

\begin{code}
\begin{minted}[frame=lines,framesep=2mm,baselinestretch=1.2,fontsize=\footnotesize]{vhdl}
-- Initialize mem32_t from hex
-- MEMORY_SIZE is IMEM_SIZE/4, see neorv32_imem.default.vhd

impure function mem32_init_hex(name : STRING; MEMORY_SIZE : natural) return mem32_t is
    file rom_file : text open read_mode is name;
    variable rom_line : line;
    variable temp_word : std_ulogic_vector(31 downto 0);
    variable temp_rom : mem32_t(0 to MEMORY_SIZE-1) := (others => (others => '0'));
begin
    for i in 0 to MEMORY_SIZE - 1 loop
        exit when endfile(rom_file);
        readline(rom_file, rom_line);
        hread(rom_line, temp_word);
        temp_rom(i) := temp_word;
    end loop;

    return temp_rom;
end function;
\end{minted}
\caption{Función a añadir al \textit{neorve32\_package.vhd} para leer un software compilado en formato hexadecimal.}
\label{code:2}
\end{code}

Además, se debe modificar el archivo \textit{neorv32\_imem.default.vhd} \footnote{En el archivo \textit{neorv32\_imem.default.vhd} se debe comentar el código relacionado con cargar la ROM desde la app\_image.} para cargar el contenido del archivo hexadecimal (\textit{neorv32\_raw\_exe.hex}) a la memoria de instrucciones, usando la función definida en el extracto de código \ref{code:2}.
Para ello se debe añadir el extracto de código \ref{code:3}.

\begin{code}
\begin{minted}[frame=lines,framesep=2mm,baselinestretch=1.2,fontsize=\footnotesize]{vhdl}
constant ROM_INIT_FILE : string := "neorv32_raw_exe.hex";
-- ROM - initialized with hex code --
constant mem_rom_c : mem32_t(0 to IMEM_SIZE/4-1) := mem32_init_hex(ROM_INIT_FILE, 
IMEM_SIZE/4);
\end{minted}
\caption{Modificación del archivo \textit{neorv32\_imem.default.vhd} para cargar la IMEM mediante la función descrita en el extracto de código \ref{code:2}.}
\label{code:3}
\end{code}

Este método propone leer desde VHDL un formato hexadecimal, el cual es una salida nativa del compilador, en lugar de autogenerar código HDL con el programa compilado como pasa cuando utilizamos la opción de la app\_image. 
Ambas opciones cargan la IMEM cuando se sintetiza el diseño, pero la opción de lectura del archivo .hex no modifica el código HDL del diseño. 
En conclusión, con esta opción conseguimos dos cosas, no autogenerar código HDL tras la compilación y no modificar el código HDL existente cada vez que se actualiza el software a cargar.

Por último, cabe destacar que a lo largo del desarrollo de este proyecto se ha cargado software compilado al NEORV32 mediante los tres formatos expuestos.
No obstante, mayoritariamente se ha utilizado el formato .vhd generando una \textit{neorv32   \_application\_image.vhd} para cada software empleado.

\section{Caracterización del rendimiento}

\label{Carac}

%Se han hecho x ensayos relatarlos, x implementaciones. 

Para caracterizar el rendimiento de los diferentes modos de conexión con los que cuenta el NEORV32, se propone acoplarle 3 tipos de multiplicadores con diferentes características.
Además, se decide realizar dos tipos de ensayos, con objeto de evaluar los métodos de conexión en términos de latencia y \textit{throughput}.
Cabe destacar que para el caso de los métodos \textit{Stream Link Interface} (SLINK) y \textit{Processor-External Bus Interface} (XBUS), se realiza una caracterización adicional acoplando los 3 multiplicadores a \textit{Verification Components}\footnote{Herramienta de verificación funcional que ofrece el \textit{framework} VUnit.} de AXI-Stream y Wishbone respectivamente.
De esta manera, se realiza por cada multiplicador acoplado mediante cada método de conexión un ensayo de latencia y si es posible de \textit{throughput}.
Esto es debido a que para realizar una caracterización de \textit{throughput}, el acelerador o el modo de conexión debe disponer de un \textit{buffer} de datos.
Teniendo en cuenta estas consideraciones, se han llevado a cabo un total de 29 ensayos de simulación con exito, todos ellos realizados mediante el \textit{framework} VUnit.
Dichos ensayos se resumen en la tabla \ref{tab:3}.
Asimismo, se han implementado en FPGA todos los diseños realizados referentes al conjunto NEORV32 más multiplicador, con objeto de verificar en placa su correcta operatividad.

\begin{table}[h!]
\centering
\caption{Ensayos de latencia y \textit{throughput} realizados: VC, el multiplicador solo acoplado a \textit{Verification Components}; C, el SoC completo incluyendo el NEORV32, el multiplicador y la ejecución de software.}
\label{tab:3}
\begin{tabular}{|cl|cc|cc|c|c|}
\hline
\multicolumn{2}{|c|}{\multirow{2}{*}{\diagbox[]{\textbf{Tipo}}{\textbf{Modo}}}} & \multicolumn{2}{c|}{\textbf{SLINK}}           & \multicolumn{2}{c|}{\textbf{XBUS}}            & \textbf{CFU} & \textbf{CFS} \\ \cline{3-8} 
\multicolumn{2}{|c|}{}                                                          & \multicolumn{1}{c|}{\textbf{VC}} & \textbf{C} & \multicolumn{1}{c|}{\textbf{VC}} & \textbf{C} & \textbf{C}   & \textbf{C}   \\ \hline
\multicolumn{2}{|c|}{\textbf{Mult-B}}                                           & \multicolumn{1}{c|}{Ambos\tablefootnote{\say{Ambos} se refiere a que se han realizado los ensayos tanto de latencia como de \textit{throughput}.}}       & Ambos      & \multicolumn{1}{c|}{Ambos}       & Ambos      & Latencia     & Ambos        \\ \hline
\multicolumn{2}{|c|}{\textbf{Mult-BP}}                                          & \multicolumn{1}{c|}{Ambos}       & Ambos      & \multicolumn{1}{c|}{Ambos}       & Ambos      & Latencia     & Ambos        \\ \hline
\multicolumn{2}{|c|}{\textbf{Mult-UBP}}                                         & \multicolumn{1}{c|}{Latencia}    & Ambos      & \multicolumn{1}{c|}{Latencia}    & Latencia   & Latencia     & Latencia     \\ \hline
\end{tabular}
\end{table}

\subsection{Descripción y conexión de los multiplicadores}

\label{decrip}

%descripcion de los multiplicadores las funciones utilizadas

\begin{figure}[h!]
    \centering
    \includegraphics[width=14cm]{Figuras/scheme.pdf}
    \caption{Esquema de las posibles combinaciones del SoC personalizado.}
    \label{fig:soc}
\end{figure}

La figura \ref{fig:soc} ilustra las posibles combinaciones de acoplamiento de los 3 diferentes multiplicadores mediante los modos de conexión SLINK, XBUS, CFU y CFS.
Las características que definen a cada tipo de multiplicador son las siguientes:

\begin{itemize}
    \item \textbf{Mult-B} (\textbf{Mult}iplicador\textit{\textbf{-B}uffered}), en el código HDL del proyecto referido como \textit{Mult\_wfifos}, es un multiplicador al que se le han añadido dos FIFOs, una a la entrada y otra a la salida, además las señales internas se gestionan mediante una máquina de estados. 
Es configurable en número de bits de entrada/salida y en profundidad de la FIFO, aunque normalmente se ha configurado en 32 bits y 4 bits respectivamente.
Puede recibir hasta tres relojes diferentes, uno en la FIFO de entrada, otro en el multiplicador y otro en la FIFO de salida, aunque comúnmente se han ajustado los tres la misma frecuencia.
Su topología interna se puede observar en el apéndice \ref{Planos}, plano \ref{fig:mult-b}.
Además, su descripción hardware se encuentra en el apéndice \ref{Codigo}, código \ref{ap-cod:3}.
\item \textbf{Mult-BP} (\textbf{Mult}iplicador\textit{\textbf{-B}uffered} y \textit{\textbf{P}ipelined}), en el código HDL del proyecto referido como \textit{Multp\_wfifos}, es un multiplicador al que se le han añadido dos FIFOs, una a la entrada y otra a la salida, pero al contrario que Mult-B, las señales internas se gestionan \textit{pipelaineadas}.
Es configurable en número de bits de entrada/salida y en profundidad de la FIFO, aunque normalmente se ha configurado en 32 bits y 4 bits respectivamente.
Puede recibir hasta tres relojes diferentes, uno en la FIFO de entrada, otro en el multiplicador y otro en la FIFO de salida, aunque comúnmente se han ajustado a la misma frecuencia.
Cabe destacar que su descripción hardware se encuentra en el apéndice \ref{Codigo}, código \ref{ap-cod:7}.
\item \textbf{Mult-UBP} (\textbf{Mult}iplicador\textit{\textbf{-U}n\textbf{B}uffered} y \textit{\textbf{P}ipelined}), en el código HDL del proyecto referido como \textit{Multp}, es un multiplicador sin \textit{buffers} de entrada/salida, además las señales internas se gestionan \textit{pipelaineadas}.
Es configurable en número de bits de entrada/salida, aunque normalmente se ha configurado en 32 bits.
Cabe destacar que su descripción hardware se encuentra en el apéndice \ref{Codigo}, código \ref{ap-cod:6}.
\end{itemize}

Para realizar el acoplamiento de los multiplicadores con las interfaces SLINK (AXI-Stream) y XBUS (Wishbone), se han realizado \textit{wrappers} definidos en el apéndice \ref{Codigo}.
En concreto, para el caso del Mult-B la descripción HDL del \textit{wrapper} se encuentra en el código \ref{ap-cod:4} y \ref{ap-cod:5}, para AXI y Wishbone respectivamente.
Para el caso del Mult-BP, en el código \ref{ap-cod:8} y \ref{ap-cod:9}, para AXI y Wishbone respectivamente.
Para el caso del Mult-UBP las señales AXI estan autocontenidas en la descripción de su diseño \ref{ap-cod:6}, así como el \textit{wrapper} para Wishbone está descrito en \ref{ap-cod:10}.

En lo que respecta al software, se han empleado funciones de C propias del NEORV32 para interactuar con cada uno de los modos de conexión.
Mediante las librerías aportadas por Stephan, tras compilarse, estas funciones se transforman a instrucciones interpretables por RISC-V.
Dichas funciones son las siguientes:
\begin{itemize}
    \item SLINK:
        \begin{itemize}
            \item Para enviar datos del NEORV32 al coprocesador: \mintinline[bgcolor=pink,breaklines]{c}{neorv32_slink_put(dato)} 
            \item Para recibir datos del coprocesador al NEORV32: \mintinline[bgcolor=pink,breaklines]{c}{neorv32_slink_get()} 
        \end{itemize}
    \item XBUS:
        \begin{itemize}
            \item Para enviar datos del NEORV32 al coprocesador: \mintinline[bgcolor=pink,breaklines]{c}{neorv32_cpu_store_}    \mintinline[bgcolor=pink,breaklines]{c}{unsigned_word(dirección,dato)} 
            \item Para recibir datos del coprocesador al NEORV32: \mintinline[bgcolor=pink,breaklines]{c}{neorv32_cpu_load_}   \mintinline[bgcolor=pink,breaklines]{c}{unsigned_word(dirección)} 
        \end{itemize}
    \item CFU:
        \begin{itemize}
            \item Para enviar datos del NEORV32 al coprocesador y recibir, en función de dichos datos, la salida del coprocesador en la misma ejecución de la instrucción: \mintinline[bgcolor=pink,breaklines]{c}{neorv32_cfu_r3_instr(funct7,funct3, rs1, rs2)} 
        \end{itemize}
    \item CFS:
        \begin{itemize}
            \item Para enviar datos del NEORV32 al registro mapeado en memoria con el coprocesador: \mintinline[bgcolor=pink,breaklines]{c}{NEORV32_CFS->REG[0] = dato_de_salida;} \footnote{Los registros disponibles asociados a CFS son del REG[0] al REG[63].}
            \item Para recibir datos del registro mapeado en memoria con el coprocesador al NEORV32: \mintinline[bgcolor=pink,breaklines]{c}{dato_de_entrada = NEORV32_CFS->REG[0];} 
        \end{itemize}
\end{itemize}

En este sentido, en el apéndice \ref{Codigo} se muestran los programas en C realizados para llevar a cabo los diferentes métodos de conexión.
Se observa que en todos ellos se emplea un \mintinline[breaklines]{c}{#ifdef}.
El objetivo de esto es separar la compilación de software orientado a ser simulado, con muy pocos \mintinline[breaklines]{c}{printf} para agilizar la simulación y la del software orientado a ser implementado en FPGA.
En concreto, en el código \ref{ap-cod:11} se muestra el main.c para SLINK, en \ref{ap-cod:12} para XBUS, en \ref{ap-cod:13} para CFU y en \ref{ap-cod:14} y \ref{ap-cod:15} para CFS.
Respecto a este último, se realiza esta división porque el primer programa está orientado a multiplicadores \textit{buffered} y el segundo al \textit{unbuffered}.
Esto es debido a que para el primer caso se destina un registro mapeado en memoria para gestionar las señales de lectura/escritura, además de otro registro para las entradas/salidas. No obstante, para el caso del multiplicador \textit{unbuffered} solo se necesita un registro para gestionar las entradas/salidas, por lo que el software varía.
Cabe destacar que se genera una app\_image por cada programa compilado, teniendo en cuenta que los \mintinline[breaklines]{c}{#define} se van comentando para obtener un programa destinado a simulación, de latencia o throughput o a implementación.

Con respecto a la CFU, se realiza una instrucción \textit{custom} para operar cada multiplicador.
En concreto, se utilizan tres instrucciones \textit{R3-Type} de la extensión Zxcfu.
Se emplea el registro \textit{rs1} para la entrada de la multiplicación, los 16 primeros bits para el primer factor y los 16 últimos para el segundo.
El segundo registro fuente \textit{rs2} no se utiliza y el resultado de la multiplicación se guarda en el registro de destino \textit{rd}.
Para especificar la instrucción de cada multiplicador se emplea el campo \textit{funct3}, también se puede emplear el campo \textit{funct7} pero en este caso no se tiene en cuenta.
En concreto, \mintinline[breaklines]{c}{funct3=000} se asocia a Mult-B, \mintinline[breaklines]{c}{funct3=001} a Mult-BP y \mintinline[breaklines]{c}{funct3=010} a Mult-UBP.
En el código \ref{ap-cod:16}, se observa como están integrados los multiplicadores en el \textit{core} del NEORV32.
Además, se aprecia como se evalúa el campo \textit{funct3} para dirigir la información de entrada/salida a cada uno de los multiplicadores, así como se observan los recursos lógicos empleados para iniciar las operaciones e indicar cuando estas han terminado.

Con respecto al subsistema CFS, como se ha mencionado, se distinguen dos casos. 
Por un lado, para el caso de los multiplicadores \textit{buffered}, se mapean 2 de los 64 registros asociados, el registro REG[0] para los datos de entrada/salida y el registro REG[1] para las señales de control.
Además, puesto que las señales de entrada/salida del subsistema CFS son ajustables, se emplean 34 bits para su salida.
En concreto, se utilizan los 2 bits MSB para realizar el control del multiplicador, es decir, gestionar sus señales de lectura/escritura.
Los otros 32 bits se emplean para representar el dato de entrada al multiplicador.
En lo que respecta a la entrada al subsistema CFS, se emplean 32 bits, es decir, el tamaño de la salida del multiplicador.
En el código \ref{ap-cod:17}, se muestra el subsistema CFS para el caso de los multiplicadores \textit{buffered}.
Por otro lado, para el multiplicador \textit{unbuffered}, tan solo se mapea el registro REG[0] con objeto de gestionar los datos de entrada/salida.
En este caso, las señales del subsistema CFS se ajustan a 32 bits tanto para la entrada como para la salida, ya que no es necesario administrar señales de control.
En el código \ref{ap-cod:18}, se muestra el subsistema CFS para el caso del multiplicador \textit{unbuffered}.

Cabe destacar que para los casos de SLINK y XBUS, simplemente se enlazan los \textit{wrappers} con el NEORV32 en el TOP del diseño. 
Es decir, no es necesario modificar archivos internos del NEORV32, como en el caso de CFU/CFS.
En el caso de XBUS, es de interés señalar que se asocia la dirección de memoria \textit{0x90000000} con dicha interfaz. 

\subsection{Metodología de medición mediante el registro CSR(mcycle)}

\label{met}

%Activada la extensión Zicntr 

%Nombrar issue k se aumenta un ciclo, quizá alguna foto de ello

La metodología de medición seguida para caracterizar procesos referentes al NEO-RV32 o al conjunto NEORV32 más coprocesador, se ha generalizado para todos los ensayos de simulación.
A continuación, se procede a explicar dicha metodología.

Cada operación realizada por el NEORV32 está asociada a una o varias instrucciones de RISC-V.
Con objeto de caracterizar una operación, se propone medir el tiempo de ejecución dedicado a computar las instrucciones que conllevan aplicar dicha operación.
Para ello, se decide emplear el registro CSR \textit{mcycle}.
Este registro se incrementa con cada ciclo de reloj activo de la CPU.
El acceso de este registro es tanto de lectura como de escritura.
Este hecho permite inicializar el CSR \textit{mcycle} a cero justo antes de comenzar el proceso a medir y leerlo en el momento que este finalice.
De esta manera, se caracteriza de forma precisa los ciclos de reloj que emplea la CPU para llevar a cabo un proceso concreto.
A pesar de que este método podría emplearse a nivel ensamblador, existen funciones en C que permiten la lectura y escritura de este registro desde un programa de alto nivel.
Estas funciones son \mintinline[breaklines]{c}{neorv32_cpu_csr_write(CSR_MCYCLE, 0)} para inicializar a cero el registro y \mintinline[breaklines]{c}{neorv32_cpu_csr_read(CSR_MCYCLE)} para leer su contenido.
De esta manera, se da la posibilidad de generar un programa en C que permita caracterizar el tiempo de ejecución de una función o funciones en C, simplemente aplicando el esquema mostrado en el extracto de código \ref{code:4}.

\begin{code}
\begin{minted}[frame=lines,framesep=2mm,baselinestretch=1.2,fontsize=\footnotesize]{c}
neorv32_cpu_csr_write(CSR_MCYCLE, 0)
//Ubicar aquí la función (o funciones) a caracterizar
neorv32_cpu_csr_read(CSR_MCYCLE)
\end{minted}
\caption{Esquema para caracterizar el tiempo de ejecución de una función (o funciones) en C.}
\label{code:4}
\end{code}

De este modo, si tenemos un programa que utilice este esquema, compilado y cargado dentro de la IMEM de un NEORV32 corriendo en una simulación, se puede extraer el valor de la medición (contenido en el registro \textit{mcycle}) y visualizarlo como resultado de la misma.
Para ello, se propone añadir el extracto de código VHDL \ref{code:5} en un test bench de VUnit.

\begin{code}
\begin{minted}[frame=lines,framesep=2mm,baselinestretch=1.2,fontsize=\footnotesize,breaklines]{vhdl}
for x in 0 to test_items-1 loop
    wait until rising_edge(clk) and csr_we = '0' and csr_valid = '1' and csr_addr = x"B00" and csr_rdata_o /= x"00000000"; -- CSR MCYCLE ADDR IS 0xB00
    info(logger, "Data " & to_string(x+1) & "/" & to_string(test_items) & " latency is " & to_string(to_integer(unsigned(csr_rdata_o))-1) & " cycles");
end loop;
\end{minted}
\caption{Código VHDL para extraer en simulación el contenido del CSR(\textit{mcycle}).}
\label{code:5}
\end{code}

Atendiendo al código \ref{code:5}, se observa que al extraer el valor del registro \textit{mcycle} se le resta un ciclo.
Esto es debido a que la ejecución de la instrucción de lectura del CSR (\textit{csrr} en ensamblador), añade un ciclo extra a la medida.
Esta situación se discutió y verificó con Stephan en una \textit{issue} titulada \href{https://github.com/stnolting/neorv32/issues/897}{\textit{Latency measurement through CSR(MCYCLE) adds one extra cycle \#897}}.
Además, el hecho de utilizar la función de VUnit \mintinline[breaklines]{vhdl}{info()} permite exportar los resultados en formato csv para su posterior procesamiento.
Cabe destacar que para emplear esta metodología debe estar activada la extensión Zicntr:

\hspace{32mm} \mintinline[breaklines]{vhdl}{CPU_EXTENSION_RISCV_Zicntr => true} 

Para ejemplificar esta explicación se dispone de uno de los \textit{test bench} de VUnit utilizados que aplican este metodología, en el apéndice \ref{Codigo} código \ref{ap-cod:19}.
En él se puede observar como se utiliza el recurso \textit{external names} para acceder a las señales de jerarquía inferior, entre ellas a las referentes a los registros CSR, con objeto de realizar su evaluación.
En concreto, este código está diseñado para caracterizar en términos de latencia el rendimiento del método de conexión CFU.
No obstante, se ha empleado este esquema para caracterizar todos los métodos de acoplamiento.

%Extraer este dato en simulación y visualizarlo en los resultados de la misma.
%Para ello se añade el codigo VHDL al test bench.
%
%
%En concreto, para realizar una transmisión cada método de conexión esta asociada a una función en C, como se ha descrito en la subsección \ref{decrip}.
%En este sentido, se propone medir el tiempo involucrado en computar las instrucciones RISC-V implicadas tras compilar dichas funciones.

\subsubsection{Descripción de los ensayos realizados}

El objetivo de esta sección \ref{Carac} es caracterizar el rendimiento de los cuatro métodos principales de conexión que ofrece el NEORV32.
Todos estos métodos están asociados a una o varias funciones en C para realizar una transmisión, como se ha descrito en la subsección \ref{decrip}.
Haciendo uso de esta metodología descrita en \ref{met}, se proponen dos tipos de ensayos.
El primero de ellos es un ensayo de latencia.
En él se realizan 4 operaciones consecutivas de envío/recepción de datos entre el NEORV32 y el multiplicador mediante cada método de conexión y se mide el tiempo de ejecución de cada una de estas operaciones en ciclos de reloj del sistema.
El segundo ensayo propuesto es de \textit{throughput}.
En él se realizan 4 operaciones de envío de datos consecutivas del NEORV32 al multiplicador, después se realizan 4 operaciones de recepción consecutivas desde el multiplicador al NEORV32 y se mide cuantos datos por ciclo de reloj se reciben.
Con objeto de almacenar los primeros 4 datos enviados, el multiplicador o el método de conexión debe contar con un \textit{buffer} de datos.
Es por ello que para el multiplicador Mult-UBP solo se puede realizar el ensayo de \textit{throughput} para el método SLINK, debido a que esta interfaz cuenta con FIFOs asociadas.
Para el caso del método CFU, solo se puede realizar la medición de latencia, ya que debido a las características internas de la instrucción personalizada, la operación de envío y recepción se realiza en un único paso.
En la figura \ref{fig:lat-thr} se aclara gráficamente estos dos tipos de ensayos.

%poner "medida" en las flechas azules
\begin{figure}[h!]
    \centering
    \includegraphics[width=14cm]{Figuras/wave_process.pdf}
    \caption{Aclaración gráfica de los dos tipos de ensayos: latencia y \textit{throughput}.}
    \label{fig:lat-thr}
\end{figure}

A este respecto, la parte de código destinada a simulación descrita en \ref{ap-cod:11}, \ref{ap-cod:12}, \ref{ap-cod:13} y \ref{ap-cod:12} para SLINK, XBUS, CFU Y CFS respectivamente, así como los respectivos \textit{test bench} de VUnit, se encargan de llevar a cabo estos ensayos.

Por último, cabe destacar que la evaluación mediante este método solo es posible para los ensayos del SoC completo, NEORV32 más multiplicador.
Para los ensayos mediante \textit{Verification Componets} se han empleado simulaciones de VUnit en conjunto con programas de Python encargados de gestionar los csv de salida producidos por la función \mintinline[breaklines]{vhdl}{info()} y así hacer los cálculos necesarios para obtener la latencia/\textit{throughput}.
En concreto, en el apéndice \ref{Codigo} se ejemplifica este proceso para la latencia de Mult-B acoplado mediante AXI-Stream.
Para ello se muestra el archivo que importa los VCs y extrae la información \ref{ap-cod:20}, el \textit{test bench} de VUnit \ref{ap-cod:21} y el \textit{script} de Python \ref{ap-cod:22} encargado de calcular a latencia.

\subsection{Resultados de los ensayos de simulación}

%Tabla & Imagenes de integración continua
%Codigo de algun test bench de VUnit
%Se muestran los resultados en simulación obtenidos en integración continua en las imágenes.

Como se ha comentado en el apartado \ref{Workf}, la realización de las simulaciones está automatizada mediante la integración continua del repositorio.
En este sentido los 29 ensayos de simulación recogidos en la tabla \ref{tab:3} se han realizado con éxito y sus resultados se muestran de la figura \ref{fig:lat1} a la \ref{fig:thr9}.
Además, con objeto de facilitar la visualización estos resultados se resumen en la tabla \ref{tab:4}.

\begin{figure}[H]
    \centering
    \includegraphics[width=14cm]{Figuras/result/lat1.png}
    \caption{Resultados del ensayo de latencia para Mult-B, Mult-BP y Mult-UBP acoplados mediante \textit{AXI-Stream Verification Componets}.}
    \label{fig:lat1}
\end{figure}

\begin{figure}[H]
    \centering
    \includegraphics[width=14cm]{Figuras/result/lat2.png}
    \caption{Resultados del ensayo de latencia para NEORV32 + Mult-B, acoplado mediante SLINK.}
    \label{fig:lat2}
\end{figure}

\begin{figure}[H]
    \centering
    \includegraphics[width=14cm]{Figuras/result/lat3.png}
    \caption{Resultados del ensayo de latencia para NEORV32 + Mult-BP, acoplado mediante SLINK.}
    \label{fig:lat3}
\end{figure}

\begin{figure}[H]
    \centering
    \includegraphics[width=14cm]{Figuras/result/lat4.png}
    \caption{Resultados del ensayo de latencia para NEORV32 + Mult-UBP, acoplado mediante SLINK.}
    \label{fig:lat4}
\end{figure}

\begin{figure}[H]
    \centering
    \includegraphics[width=14cm]{Figuras/result/thr1.png}
    \caption{Resultados del ensayo de \textit{throughput} para Mult-B, Mult-BP acoplados mediante \textit{AXI-Stream Verification Componets}.}
    \label{fig:thr1}
\end{figure}

\begin{figure}[H]
    \centering
    \includegraphics[width=14cm]{Figuras/result/thr2.png}
    \caption{Resultados del ensayo de \textit{throughput} para NEORV32 + Mult-B, acoplado mediante SLINK.}
    \label{fig:thr2}
\end{figure}

\begin{figure}[H]
    \centering
    \includegraphics[width=14cm]{Figuras/result/thr3.png}
    \caption{Resultados del ensayo de \textit{throughput} para NEORV32 + Mult-BP, acoplado mediante SLINK.}
    \label{fig:thr3}
\end{figure}

\begin{figure}[H]
    \centering
    \includegraphics[width=14cm]{Figuras/result/thr4.png}
    \caption{Resultados del ensayo de \textit{throughput} para NEORV32 + Mult-UBP, acoplado mediante SLINK.}
    \label{fig:thr4}
\end{figure}

\begin{figure}[H]
    \centering
    \includegraphics[width=14cm]{Figuras/result/lat5.png}
    \caption{Resultados del ensayo de latencia para Mult-B, Mult-BP y Mult-UBP acoplados mediante \textit{Wishbone Verification Componets}.}
    \label{fig:lat5}
\end{figure}

\begin{figure}[H]
    \centering
    \includegraphics[width=14cm]{Figuras/result/lat6.png}
    \caption{Resultados del ensayo de latencia para NEORV32 + Mult-B, acoplado mediante XBUS.}
    \label{fig:lat6}
\end{figure}

\begin{figure}[H]
    \centering
    \includegraphics[width=14cm]{Figuras/result/lat7.png}
    \caption{Resultados del ensayo de latencia para NEORV32 + Mult-BP, acoplado mediante XBUS.}
    \label{fig:lat7}
\end{figure}

\begin{figure}[H]
    \centering
    \includegraphics[width=14cm]{Figuras/result/lat8.png}
    \caption{Resultados del ensayo de latencia para NEORV32 + Mult-UBP, acoplado mediante XBUS.}
    \label{fig:lat8}
\end{figure}

\begin{figure}[H]
    \centering
    \includegraphics[width=14cm]{Figuras/result/thr5.png}
    \caption{Resultados del ensayo de \textit{throughput} para Mult-B, Mult-BP acoplados mediante \textit{Wishbone Verification Componets}.}
    \label{fig:thr5}
\end{figure}

\begin{figure}[H]
    \centering
    \includegraphics[width=14cm]{Figuras/result/thr6.png}
    \caption{Resultados del ensayo de \textit{throughput} para NEORV32 + Mult-B, acoplado mediante XBUS.}
    \label{fig:thr6}
\end{figure}

\begin{figure}[H]
    \centering
    \includegraphics[width=14cm]{Figuras/result/thr7.png}
    \caption{Resultados del ensayo de \textit{throughput} para NEORV32 + Mult-BP, acoplado mediante XBUS.}
    \label{fig:thr7}
\end{figure}

\begin{figure}[H]
    \centering
    \includegraphics[width=14cm]{Figuras/result/lat9.png}
    \caption{Resultados del ensayo de latencia para NEORV32 + Mult-B, acoplado mediante CFU.}
    \label{fig:lat9}
\end{figure}

\begin{figure}[H]
    \centering
    \includegraphics[width=14cm]{Figuras/result/lat10.png}
    \caption{Resultados del ensayo de latencia para NEORV32 + Mult-BP, acoplado mediante CFU.}
    \label{fig:lat10}
\end{figure}

\begin{figure}[H]
    \centering
    \includegraphics[width=14cm]{Figuras/result/lat11.png}
    \caption{Resultados del ensayo de latencia para NEORV32 + Mult-UBP, acoplado mediante CFU.}
    \label{fig:lat11}
\end{figure}

\begin{figure}[H]
    \centering
    \includegraphics[width=14cm]{Figuras/result/lat12.png}
    \caption{Resultados del ensayo de latencia para NEORV32 + Mult-B, acoplado mediante CFS.}
    \label{fig:lat12}
\end{figure}

\begin{figure}[H]
    \centering
    \includegraphics[width=14cm]{Figuras/result/lat13.png}
    \caption{Resultados del ensayo de latencia para NEORV32 + Mult-BP, acoplado mediante CFS.}
    \label{fig:lat13}
\end{figure}

\begin{figure}[H]
    \centering
    \includegraphics[width=14cm]{Figuras/result/lat14.png}
    \caption{Resultados del ensayo de latencia para NEORV32 + Mult-UBP, acoplado mediante CFS.}
    \label{fig:lat14}
\end{figure}

\begin{figure}[H]
    \centering
    \includegraphics[width=14cm]{Figuras/result/thr8.png}
    \caption{Resultados del ensayo de \textit{throughput} para NEORV32 + Mult-B, acoplado mediante CFS.}
    \label{fig:thr8}
\end{figure}

\begin{figure}[H]
    \centering
    \includegraphics[width=14cm]{Figuras/result/thr9.png}
    \caption{Resultados del ensayo de \textit{throughput} para NEORV32 + Mult-BP, acoplado mediante CFS.}
    \label{fig:thr9}
\end{figure}

\begin{table}[h!]
\centering
\caption{Resultados de los ensayos de latencia y \textit{throughput}: VC, el multiplicador solo acoplado a \textit{Verification Components}; C, el SoC completo incluyendo el NEORV32, el multiplicador y la ejecución de software.}
\label{tab:4}
\begin{tabular}{|cc|cc|cc|c|c|}
\hline
\multicolumn{2}{|c|}{\multirow{2}{*}{\textbf{Mult}}}                                    & \multicolumn{2}{c|}{\textbf{SLINK}}           & \multicolumn{2}{c|}{\textbf{XBUS}}            & \textbf{CFU} & \textbf{CFS} \\ \cline{3-8} 
\multicolumn{2}{|c|}{}                                                                  & \multicolumn{1}{c|}{\textbf{VC}} & \textbf{C} & \multicolumn{1}{c|}{\textbf{VC}} & \textbf{C} & \textbf{C}   & \textbf{C}   \\ \hline
\multicolumn{1}{|c|}{\multirow{3}{*}{\textbf{Latencia}}}            & \textbf{Mult-B}   & \multicolumn{1}{c|}{6}           & 45         & \multicolumn{1}{c|}{5}           & 16         & 13           & 37           \\ \cline{2-8} 
\multicolumn{1}{|c|}{}                                              & \textbf{Mult-BP}  & \multicolumn{1}{c|}{4}           & 45         & \multicolumn{1}{c|}{3}           & 16         & 11           & 37           \\ \cline{2-8} 
\multicolumn{1}{|c|}{}                                              & \textbf{Mult-UBP} & \multicolumn{1}{c|}{1}           & 45         & \multicolumn{1}{c|}{2}           & 16         & 8            & 18           \\ \hline
\multicolumn{1}{|c|}{\multirow{3}{*}{\textit{\textbf{Throughput}}}} & \textbf{Mult-B}   & \multicolumn{1}{c|}{1/4}         & 1/20       & \multicolumn{1}{c|}{1/2}         & 1/5        & X            & 1/15         \\ \cline{2-8} 
\multicolumn{1}{|c|}{}                                              & \textbf{Mult-BP}  & \multicolumn{1}{c|}{1}           & 1/20       & \multicolumn{1}{c|}{1/2}         & 1/5        & X            & 1/15         \\ \cline{2-8} 
\multicolumn{1}{|c|}{}                                              & \textbf{Mult-UBP} & \multicolumn{1}{c|}{X}           & 1/20       & \multicolumn{1}{c|}{X}           & X          & X            & X            \\ \hline
\end{tabular}
\end{table}

\vspace{3cm}

Cabe destacar que por cada ensayo de simulación se ha producido un archivo .vcd que contiene las \textit{waveforms} producidas.
En este sentido, la integración continua se encarga de gestionar estos archivos y subirlos como \textit{artifacts}.
En el apéndice \ref{wave}, se expone la forma de onda referente al ensayo de \textit{throughput} para NEORV32 + Mult-BP acoplado mediante XBUS \ref{wave:xbus} , así como la referente al ensayo de latencia para NEORV32 + Mult-B acoplado mediante CFU  \ref{wave:cfu}.

\subsection{Implementación en FPGA}

% captura cutecom de la implementación.

Todos las combinaciones del SoC, reflejadas en la figura \ref{fig:soc}, se han verificado tanto en la placa Arty A7 35T como en la 100T, lo que ha supuesto la generación duplicada de los \textit{bitstreams}.
Esta generación se ha realizado mediante las dos vías expuestas en la sección \ref{Workf}, herramientas FLOS y Vivado.
En la imagen \ref{fig:impl-gh} se muestran los procesos de integración continua para generar de forma automatizada los \textit{bitstreams} mediante herramientas FLOS en el repositorio de GitHub. 
En adición a esto, en el apendice \ref{Codigo} código \ref{ap-cod:23}, se muestra un ejemplo de un archivo bash realizado para generar automáticamente el \textit{bitstream} de la implementación CFU mediante herramientas FLOS.
En él se observa un matiz destacable, al realizar la síntesis con yosys se añaden los argumentos  \mintinline[breaklines]{bash}{-nodsp} y \mintinline[breaklines]{bash}{-nolutram}.
Este hecho es debido a que la síntesis para la Arty mediante yosys no está del todo pulida, así como para Lattice si, para Xilinix temas como la gestión de DSPs y de RAM distribuida todavía no están soportados.
Además, para obtener un correcto funcionamiento de los \textit{bitstreams} generados con herramientas FLOS, se ha tenido que rebajar la capacidad de la IMEM a:

\hspace{30mm} \mintinline[breaklines]{vhdl}{MEM_INT_IMEM_SIZE : natural := 6*1024}

\noindent En adición a la generación, el proceso de integración continua también automatiza la subida de los \textit{bitstreams} como artefactos.
En consecuencia, se han descargado y testeado todos ellos en ambas placas para comprobar su correcta operatividad.
En este sentido, se ha obteniendo un resultado satisfactorio.

\begin{figure}[H]
    \centering
    \includegraphics[width=14cm]{Figuras/impl-gh.png}
    \caption{Procesos de integración continua para generar los \textit{bitstreams} de los ensayos llevados a cabo mediante herramientas FLOS.}
    \label{fig:impl-gh}
\end{figure}

Los ensayos de implementación siguen un esquema similar a los ensayos de simulación.
No obstante, la finalidad no es realizar la caracterización de rendimiento, sino verificar la correcta operatividad.
En este sentido, todos los ensayos operan los mismos datos mediante el siguiente procedimiento.
Un programa de la IMEM, resultado de la compilación de la parte de código C destinada a implementación descrita en \ref{ap-cod:11}, \ref{ap-cod:12}, \ref{ap-cod:13} y \ref{ap-cod:12}, lanza 4 datos de entrada\footnote{Los datos de entrada son 1 x 1, 2 x 2, 4 x 4 y 8 x 8 (al igual que en simulación); por lo que se debe obtener en hexadecimal 0x1, 0x4, 0x10 (16) y 0x40 (64).} a los multiplicadores mediante cada método de acoplamiento, el acelerador los multiplica y los devuelve al NEORV32 el cual los manda por UART visualizando el resultado en el ordenador.
De todas las implementaciones, realizadas se muestran cuatro con objeto de ejemplificar la correcta operatividad de los diseños.
Los resultados que se proceden a mostrar se han visualizado mediante la terminal CuteCom.
En concreto, la figura \ref{fig:impl1} refiere al multiplicador Mult-B acoplado al NEORV32 mediante SLINK, la figura \ref{fig:impl1} refiere al multiplicador Mult-BP acoplado mediante XBUS, la figura \ref{fig:impl1} refiere los tres multiplicadores acoplados mediante CFU y la figura \ref{fig:impl1} refiere al multiplicador Mult-UBP acoplado mediante CFS.

\begin{figure}[H]
    \centering
    \includegraphics[width=14cm]{Figuras/impl1.png}
    \caption{Ensayo de implementación de Mult-B acoplado al NEORV32 mediante SLINK.}
    \label{fig:impl1}
\end{figure}

\begin{figure}[H]
    \centering
    \includegraphics[width=14cm]{Figuras/impl2.png}
    \caption{Ensayo de implementación de Mult-BP acoplado al NEORV32 mediante XBUS.}
    \label{fig:impl2}
\end{figure}

\begin{figure}[H]
    \centering
    \includegraphics[width=14cm]{Figuras/impl3.png}
    \caption{Ensayo de implementación de Mult-B, Mult-BP y Mult-UBP acoplados al NEORV32 mediante CFU.}
    \label{fig:impl3}
\end{figure}

\begin{figure}[H]
    \centering
    \includegraphics[width=14cm]{Figuras/impl4.png}
    \caption{Ensayo de implementación de Mult-UBP acoplado al NEORV32 mediante CFS.}
    \label{fig:impl4}
\end{figure}

\section{Integración de coprocesador para aplicaciones de IA}

\label{Integ}

Una vez generada la caracterización de todos los métodos de conexión, se cuenta con un criterio objetivo para escoger un modo de acoplamiento y conectar de forma óptima un coprocesador para acelerar aplicaciones de IA.
Como se ha mencionado, el coprocesador implementado únicamente se encarga de calcular la activación por medio de la función sigmoide.
%nexo
El procedimiento seguido para realizar la integración de este coprocesador se ha dividido en tres etapas. 
En la primera etapa, se realiza un ensayo de simulación del acelerador en solitario, además se acopla al NEORV32 mediante el método CFU con objeto de verificar en placa la correcta operatividad del mismo.
En la segunda etapa, se busca una manera de calcular la función de activación sigmoide haciendo uso de la unidad de coma flotante que ofrece el NEORV32.
En la tercera etapa, se realiza un ensayo comparativo, tanto en simulación como en implementación, para verificar las ventajas en términos de latencia de realizar el cálculo de la FA mediante un enfoque distribuido.

\subsection{Verificación de la operatividad del acelerador}

El acelerador ha sido proporcionado por el grupo de investigación.
Como se ha mencionado en \ref{cri}, este coprocesador acelera los cálculos mediante el método de interpolación recursiva centrada.
Cabe destacar que de las 7 funciones de activación disponibles solo se ha empleado la lógica de la función sigmoide.
Además, se han debido de hacer las debidas adaptaciones para conseguir la correcta operatividad del circuito en un entorno de VUnit.

El objetivo de la verificación es sencillo, se deben recrear los resultados obtenidos por Basterretxea para los 9 datos de entrada recogidos en la tabla \ref{tab:5}.
Para ello se comienza con una simulación del coprocesador en solitario en el entorno de VUnit.
Después, se acopla al NEORV32 mediante CFU y se realiza una implementación en placa.
En este sentido, el NEORV32 se encargará de enviar los 9 datos de entrada al coprocesador, asimismo este calculará y devolverá el resultado de la FA de vuelta al microcontrolador.
Tanto el ensayo en simulación del coprocesador en solitario como la generación del \textit{bitstream} del ensayo en implementación del coprocesador acoplado al NEORV32 mediante CFU, se realizan mediante tareas automatizadas en la integración continua del repositorio de GitLab.
Estos ensayos no se encuentran disponibles en el repositorio propio de GitHub.

\begin{table}[h!]
\centering
\caption{Resultados de la función sigmoide a verificar.}
\label{tab:5}
\begin{tabular}{|c|ccc|}
\hline
\multirow{3}{*}{\textbf{Dato de entrada}} & \multicolumn{3}{c|}{\textbf{Resultado de la función sigmoide}}                                           \\ \cline{2-4} 
                                          & \multicolumn{2}{c|}{\textbf{Matlab}}                                                & \textbf{Modelsim}  \\ \cline{2-4} 
                                          & \multicolumn{1}{c|}{\textbf{Coma fija}} & \multicolumn{1}{c|}{\textbf{Hexadecimal}} & \textbf{Coma fija} \\ \hline
-10                                       & \multicolumn{1}{c|}{0}                  & \multicolumn{1}{c|}{0000}                 & 0                  \\ \hline
-7,5                                      & \multicolumn{1}{c|}{0,0078125}          & \multicolumn{1}{c|}{0002}                 & 0,0078125          \\ \hline
-5                                        & \multicolumn{1}{c|}{0,15625}            & \multicolumn{1}{c|}{0028}                 & 0,15625            \\ \hline
-2,5                                      & \multicolumn{1}{c|}{0,484375}           & \multicolumn{1}{c|}{007C}                 & 0,484375           \\ \hline
0                                         & \multicolumn{1}{c|}{1}                  & \multicolumn{1}{c|}{0100}                 & 1                  \\ \hline
2,5                                       & \multicolumn{1}{c|}{1,515625}           & \multicolumn{1}{c|}{0184}                 & 1,515625           \\ \hline
5                                         & \multicolumn{1}{c|}{1,84375}            & \multicolumn{1}{c|}{01D8}                 & 1,84375            \\ \hline
7,5                                       & \multicolumn{1}{c|}{1,9921875}          & \multicolumn{1}{c|}{01FE}                 & 1,9921875          \\ \hline
10                                        & \multicolumn{1}{c|}{2}                  & \multicolumn{1}{c|}{0200}                 & 2                  \\ \hline
\end{tabular}
\end{table}

Los resultados obtenidos en simulación del coprocesador en solitario se muestran en la figura \ref{fig:sim_alone}, así como los obtenidos en implementación para el coprocesador acoplado al NEORV32 vía CFU se muestran en la figura \ref{fig:impl_complex}.
En adición a esto, el código que describe el acelerador utilizado se encuentra en el apéndice \ref{Codigo} código \ref{ap-cod:24}, así como la integración en CFU se muestra en el código \ref{ap-cod:25}.
Además, se aclara gráficamente la integración del acelerador mediante CFU en el plano \ref{}. 
%hacer plano con latex
Por último, se muestra la forma de onda resultante de la ejecución de un dato por el acelerador sigmoide CRI mediante CFU en el apéndice, \ref{wave} forma de onda \ref{}. 
En esta forma de onda se observa algo interesante, las señales de la CFU varían con respecto a la forma de onda \ref{wave:cfu} (implementación del multiplicador).
Esto es debido a que en el transcurso de la elaboración de esta investigación la integración mediante CFU cambió ligeramente, como se ha mencionado en el apartado \ref{neorv32}.
%hacer forma de onda
Al igual que para el caso del multiplicador, se emplea una instrucción personalizada \textit{R3-Type} de la extensión Zxcfu.
Se emplea el registro fuente \textit{rs1} para la entrada del acelerador.
Ya que los datos a verificar se generaron con una longitud de palabra WL = 16 bits, se ha querido preservar el acelerador a este tamaño de entrada/salida para todos los ensayos realizados en este capítulo, consultar \ref{ap-cod:25} para observar como se ha gestionado este longitud con el tamaño de los registros \textit{rs1} y \textit{rs1} de 32 bits. 
El registro \textit{rs2} no se utiliza y el resultado del cálculo de la función sigmoide se guarda en el registro de destino \textit{rd}.
Para especificar la función se ha empleado el campo \textit{funct3}, asociando el coprocesador sigmoide a \mintinline[breaklines]{c}{funct3=000}.

%Añadir la flexibilidad de cli, se puede ajustar a distintos q + x ciclos

\begin{figure}[H]
    \centering
    \includegraphics[width=14cm]{Figuras/sim_sig_alone.png}
    \caption{Verificación en simulación de la correcta operatividad del acelerador sigmoide en un entorno de VUnit.}
    \label{fig:sim_alone}
\end{figure}

\begin{figure}[H]
    \centering
    \includegraphics[width=14cm]{Figuras/sig_cfu.png}
    \caption{Verificación en implementación de la correcta operatividad del acelerador sigmoide acoplado al NEORV32 mediante CFU.}
    \label{fig:impl_complex}
\end{figure}

Como se puede apreciar en las imágenes los resultados en ambos casos han sido los esperados, cumpliendo de forma óptima la primera etapa de verificación.

\subsection{Realización del cálculo de la FA mediante la FPU}

\label{fpu-fa}

La tarea de calcular la función de activación sigmoide mediante los recursos por defecto del NEORV32, se ha decidido implementar mediante la Unidad de Coma Flotante (FPU).
%si da tiempo hacer ensayo de mult float sin FPU
Es decir, al contrario que el acelerador que se usa coma fija, para el enfoque realizado sin coprocesador se emplean datos en coma flotante.
Normalmente, cuando una red se calcula por software, es decir, sin externalizar los cálculos a coprocesadores embebidos, los datos que se suelen usar son de este tipo, por lo que se ha visto adecuado usarlos.
En concreto, las instrucciones relativas al manejo de la Unidad de Coma Flotante se encuentran en la extensión Zfinx.
En este sentido surge un pequeño inconveniente, debido a que la implementación hardware actual de la extensión en el NEORV32 es limitada, esta \href{https://stnolting.github.io/neorv32/#_zfinx_isa_extension}{no soporta operaciones de división}.
A priori esto puede resultar un problema, ya que la función sigmoide es una fracción \eqref{ec:2}, resolverla sin emplear operaciones de división puede resultar engorroso.
Es por ello que se ha buscado una alternativa de cálculo.
En primer lugar, se ha de tener en cuenta que las operaciones de suma y de multiplicación si que están soportadas mediante las funciones \mintinline[bgcolor=pink,breaklines]{c}{riscv_intrinsic_fadds()} y \mintinline[bgcolor=pink,breaklines]{c}{riscv_intrinsic_fmuls()} respectivamente.
Este hecho nos abre una ventana de posibilidad y nos permite implementar una aproximación polinomial a la función sigmoide.
Para ello se ha realizado un programa en python descrito en el código \ref{ap-cod:26}.
Este programa hace uso de la librería numpy y su funcionamiento es la siguiente: se describe la función a aproximar, en este caso \eqref{ec:3} (sigmoide ajustada a saturación = 2 y pendiente = 0,5), se calculan los coeficientes para tres grados distintos de polinomios y se realiza una gráfica comparativa.
Cabe destacar que el rango de aproximación definido está condicionado por los valores a testear, es decir, entre -10 y 10.
Se decide realizar polinomios de grado 3\eqref{ec:4}, 5\eqref{ec:5} y 7\eqref{ec:6} con objeto de testear varías implementaciones y no comprometer ni el tiempo de ejecución ni la precisión, obteniendo varios ejemplos a comparar con el coprocesador. 

\begin{equation}\label{ec:3}
f(x)= 2 \times \frac{1}{1 + e^{-0,5x}}\tag{***}
\end{equation}

Los polinomios de aproximación a la función sigmoide obtenidos son los siguientes:

\begin{equation}\label{ec:4}
f(x)= 1 + 0.19744040439x -0.00109773200x^3\tag{****}
\end{equation}


\begin{equation}\label{ec:5}
f(x)= 1 +  0.22878851178x - 0.00253272801x^3 + 0.00001266682x^5\tag{*****}
\end{equation}

\begin{equation}\label{ec:6}
f(x)= 1 + 0.24190955524x - 0.00369209483x^3 + + 0.00003770457x^5 - 0.00000015213x^7 \tag{******}
\end{equation}

Como se observa en la figura \ref{fig:comp}, se obtienen un buen resultado de aproximación mediante las funciones polinómicas propuestas en el rango indicado.
En este sentido, en el código \ref{ap-cod:27} se puede observar como se implementan estas funciones en C.
En adición a esto, cabe destacar que el NEORV32 no soporta la impresión de flotantes por UART, por lo que se ha tenido que adaptar el resultado a dos enteros que representen la parte entera y la parte decimal.
Para este propósito, entre otras cosas, se ha utilizado la función \mintinline[bgcolor=pink,breaklines]{c}{riscv_intrinsic_flts()} para realizar la comparación \say{menor que} de un flotante con cero y evaluar su signo.
Para emplear las funciones de la extensión Zfinx, esta se debe activar en el TOP del diseño.

\hspace{32mm} \mintinline[breaklines]{vhdl}{CPU_EXTENSION_RISCV_Zfinx => true} 

\begin{figure}[H]
    \hspace*{-38mm}
    \centering
    \includegraphics[width=22cm]{Figuras/Comp_aprox.pdf}
    \caption{Comparación entre la función original y las aproximaciones polinómicas de grado 3, 5 y 7.}
    \label{fig:comp}
\end{figure}

Cabe destacar un hecho relevante, como se ha mencionado la aproximación polinómica está calculada en función del rango propuesto.
Es decir, su uso no se puede generalizar para todos los valores de entrada de la recta real.
Esto es debido a que para valores fuera de rango indicado, el polinomio se puede desviar de la aproximación a la función sigmoide.
Para evitar esto, si se desea generalizar estos polinomios, se recomienda saturar a 0 los valores inferiores a -10  y saturar a 2 los valores superiores a 10. 
Para el caso del polinomio de grado 3 se recomendaría saturar algo antes.

\subsection{Comparación de enfoques}

La comparación de ambos enfoques se ha llevado a cabo mediante una simulación de VUnit  y una implementación en placa.
Tanto la simulación, como la generación del \textit{bitstream} del ensayo en placa, se han automatizo en tareas de integración continua del repositorio de GitLab.
Estos ensayos no se encuentran disponibles en el repositorio propio de GitHub.

\subsubsection{Simulación}

El ensayo de simulación consiste en lanzar un mismo paquete de 9 datos 4 veces, un vez por cada implementación sigmoide a testear y medir los ciclos que tarda cada una de ellas en calcular el resultado para cada dato de entrada.
En este sentido, se ha acoplado mediante CFU el coprocesador CRI al NEORV32 con su FPU activada, con la cual se calculan las otras tres implementaciones.
El orden seguido es el siguiente, en primer lugar se lanza el paquete al acelerador CRI, en segundo lugar a la aproximación polinómica de grado 7, en tercer lugar a la aproximación polinómica de grado 5 y por último a la aproximación polinómica de grado 3, 36 datos en total.
Para realizar las medidas de latencia, se ha utilizado la metodología expuesta en el apartado \ref{met}.
El código empleado en esta simulación se encuentra en la parte de código destinada a simulación del programa \ref{ap-cod:27}.
Los resultados obtenidos se muestran en la figura \ref{fig:sim_comp}

\begin{figure}[H]
    \centering
    \includegraphics[width=14cm]{Figuras/SIM_RES_CI.png}
    \caption{Resultados de simulación en ciclos de latencia necesarios para calcular la función sigmoide para cada dato de entrada.}
    \label{fig:sim_comp}
\end{figure}

\subsubsection{Implementación}

El ensayo en placa ha consistido en dos partes contenidas en un mismo programa, ver la parte de código destinada a implementación en \ref{ap-cod:27}.
Por un lado, se ha computado el resultado de la función sigmoide con los dos enfoques, el distribuido mediante el coprocesador acoplado al NEORV32 y el monolítico a través de las tres aproximaciones mediante la FPU.
Para ello, se han utilizado los mismo 9 datos de entrada que en simulación.
Por otro lado, se ha repetido cada uno de estos cómputos y mediante el uso del CSR(mcycle) se ha calculado la latencia que tarda cada uno de los enfoques en calcular un paquete completo de datos, dando un punto de vista más global.
La figura \ref{fig:impl_comp} muestra el resultado del ensayo transmitido por UART y visualizado mediante la terminal CuteCom.

\begin{figure}[H]
    \centering
    \includegraphics[width=14cm]{Figuras/RES_impl_comp.png}
    \caption{Resultado del ensayo comparativo en placa.}
    \label{fig:impl_comp}
\end{figure}

\subsubsection{Resultados}

Se procede a agrupar los resultados obtenidos en dos tablas.
Una referente al resultado del cálculo de la función sigmoide \ref{tab:6} y otra a la latencia relativa a cada implementación \ref{tab:7}.

Respecto a los resultados de la sigmoide, se observa que para los valores cercanos a los límites del rango relativos a las aproximaciones de grado bajo es necesario realizar una saturación, como ya se ha destacado en la subsección \ref{fpu-fa}.
No obstante, la aproximación de grado más alto se ajusta con relativa eficacia a los valores esperados.
Respecto a la latencia, cabe destacar que la latencia mínima para los casos de las aproximaciones mediante FPU coincide siempre con el 5º dato, el cual es f(0) = 1.
De este hecho se deduce que se requiere poca carga computacional para calcular este dato, es por ello que baja tanto la latencia mínima.
Si se observa el resto de casos, figura \ref{fig:sim_comp}, la desviación no es tan notable.
Respecto a los resultados del paquete de datos se observa que no se ajusta de forma directa al total de ciclos entre 9.
Esto es debido a que en el ensayo de simulación se mide rigurosamente cada ejecución de dato aislada, mientras que la implementación en placa mide, aparte de la ejecución per se, las instrucciones relativas al bucle for que corre la secuencia de introducción del paquete, lo que añade ciclos extra a la computación, ver \ref{ap-cod:27}.
Es por ello que se ha mencionado que el ensayo en placa da un punto de vista más global, ya que en la práctica no solo se ejecutan los ciclos relativos a las instrucciones propias, siempre van acompañadas de código extra que compone la aplicación.
Respecto al ahorro computacional en ciclos de latencia obtenidos mediante el acelerador CRI, podemos afirmar que en el caso más ajustado se ahorran 111 - 13 = 98 ciclos, lo cual es una aceleración del 8,53 y en el caso más holgado se ahorran 280 - 13 = 267 ciclos, lo cual es una aceleración del 20,53.
Si realizamos la media de las tres medias y comparamos obtenemos un ahorro de 198,3 - 13 = 185,3 ciclos, lo cual es una aceleración del 14,25.
Estos hechos corroboran el beneficio de emplear este tipo de enfoque distribuido y afianza el argumento de externalizar los cálculos relativos a las redes neuronales a coprocesadores embebidos con la finalidad de llevar a cabo una gestión eficaz y simplificada de IA en el borde.


%tabla de resultados

\begin{table}[H]
\centering
\caption{Resultados de la función sigmoide obtenidos mediante los cuatro casos ensayados (implementación).}
\label{tab:6}
\begin{tabular}{|c|cc|ccc|}
\hline
\multirow{3}{*}{\textbf{Entrada}} & \multicolumn{2}{c|}{\multirow{2}{*}{\textbf{\begin{tabular}[c]{@{}c@{}}Coprocesador\\ CRI\end{tabular}}}} & \multicolumn{3}{c|}{\textbf{Aproximación mediante FPU}}                                          \\ \cline{4-6} 
                                  & \multicolumn{2}{c|}{}                                                                                     & \multicolumn{1}{c|}{\textbf{Grado 7}} & \multicolumn{1}{c|}{\textbf{Grado 5}} & \textbf{Grado 3} \\ \cline{2-6} 
                                  & \multicolumn{1}{c|}{\textbf{Coma fija}}                       & \textbf{Hexadecimal}                      & \multicolumn{3}{c|}{\textbf{Coma flotante}}                                                      \\ \hline
-10                               & \multicolumn{1}{c|}{0}                                        & 0000                                      & \multicolumn{1}{c|}{0,023842692}      & \multicolumn{1}{c|}{-0,021839262}     & 0,123327968      \\ \hline
-7,5                              & \multicolumn{1}{c|}{0,0078125}                                & 0002                                      & \multicolumn{1}{c|}{0,051602840}      & \multicolumn{1}{c|}{0,051991312}      & -0,017697334     \\ \hline
-5                                & \multicolumn{1}{c|}{0,15625}                                  & 0028                                      & \multicolumn{1}{c|}{0,146022496}      & \multicolumn{1}{c|}{0,133064584}      & 0,150014512      \\ \hline
-2,5                              & \multicolumn{1}{c|}{0,484375}                                 & 007C                                      & \multicolumn{1}{c|}{0,449325888}      & \multicolumn{1}{c|}{0,466365568}      & 0,523551104      \\ \hline
0                                 & \multicolumn{1}{c|}{1}                                        & 0100                                      & \multicolumn{1}{c|}{1,00}             & \multicolumn{1}{c|}{1,00}             & 1,00             \\ \hline
2,5                               & \multicolumn{1}{c|}{1,515625}                                 & 0184                                      & \multicolumn{1}{c|}{1,550674176}      & \multicolumn{1}{c|}{1,533634432}      & 1,476448896      \\ \hline
5                                 & \multicolumn{1}{c|}{1,84375}                                  & 01D8                                      & \multicolumn{1}{c|}{1,853977536}      & \multicolumn{1}{c|}{1,866935360}      & 1,849985472      \\ \hline
7,5                               & \multicolumn{1}{c|}{1,9921875}                                & 01FE                                      & \multicolumn{1}{c|}{1,948397184}      & \multicolumn{1}{c|}{1,948008640}      & 2,017697334      \\ \hline
10                                & \multicolumn{1}{c|}{2}                                        & 0200                                      & \multicolumn{1}{c|}{1,976157312}      & \multicolumn{1}{c|}{2,021839142}      & 1,876671936      \\ \hline
\end{tabular}
\end{table}

%tabla latencias

\begin{table}[H]
\centering
\caption{Resultados de latencia en ciclos de reloj del sistema obtenidos para los cuatro casos ensayados (simulación/implementación).}
\label{tab:7}
\begin{tabular}{|cccccccccccc|}
\hline
\multicolumn{12}{|c|}{\textbf{Un dato (simulación)}}                                                                                                                                                                                                                                                                                                                                                                             \\ \hline
\multicolumn{3}{|c|}{\multirow{2}{*}{\textbf{\begin{tabular}[c]{@{}c@{}}Coprocesador\\ CRI\end{tabular}}}} & \multicolumn{9}{c|}{\textbf{Aproximación mediante FPU}}                                                                                                                                                                                                                                                \\ \cline{4-12} 
\multicolumn{3}{|c|}{}                                                                                     & \multicolumn{3}{c|}{\textbf{Grado 7}}                                                                   & \multicolumn{3}{c|}{\textbf{Grado 5}}                                                                   & \multicolumn{3}{c|}{\textbf{Grado 3}}                                              \\ \hline
\multicolumn{1}{|c|}{\textbf{Max}}  & \multicolumn{1}{c|}{\textbf{M\tablefootnote{\say{M} refiere a la media.}}}  & \multicolumn{1}{c|}{\textbf{Min}} & \multicolumn{1}{c|}{\textbf{Max}} & \multicolumn{1}{c|}{\textbf{M}} & \multicolumn{1}{c|}{\textbf{Min}} & \multicolumn{1}{c|}{\textbf{Max}} & \multicolumn{1}{c|}{\textbf{M}} & \multicolumn{1}{c|}{\textbf{Min}} & \multicolumn{1}{c|}{\textbf{Max}} & \multicolumn{1}{c|}{\textbf{M}} & \textbf{Min} \\ \hline
\multicolumn{1}{|c|}{13}            & \multicolumn{1}{c|}{13}          & \multicolumn{1}{c|}{13}           & \multicolumn{1}{c|}{280}          & \multicolumn{1}{c|}{259,1}      & \multicolumn{1}{c|}{203}          & \multicolumn{1}{c|}{210}          & \multicolumn{1}{c|}{198,5}      & \multicolumn{1}{c|}{157}          & \multicolumn{1}{c|}{146}          & \multicolumn{1}{c|}{137,5}      & 111          \\ \hline
\multicolumn{12}{|c|}{\textbf{Un paquete de 9 datos (implementación)}}                                                                                                                                                                                                                                                                                                                                                                          \\ \hline
\multicolumn{3}{|c|}{\multirow{2}{*}{\textbf{\begin{tabular}[c]{@{}c@{}}Coprocesador\\ CRI\end{tabular}}}} & \multicolumn{9}{c|}{\textbf{Aproximación mediante FPU}}                                                                                                                                                                                                                                                \\ \cline{4-12} 
\multicolumn{3}{|c|}{}                                                                                     & \multicolumn{3}{c|}{\textbf{Grado 7}}                                                                   & \multicolumn{3}{c|}{\textbf{Grado 5}}                                                                   & \multicolumn{3}{c|}{\textbf{Grado 3}}                                              \\ \hline
\multicolumn{3}{|c|}{197}                                                                                  & \multicolumn{3}{c|}{2424}                                                                               & \multicolumn{3}{c|}{1875}                                                                               & \multicolumn{3}{c|}{1524}                                                          \\ \hline
\end{tabular}
\end{table}



\chapter{Metodología seguida en el desarrollo del trabajo} % Título del capítulo

\label{Metodología} % etiqueta \ref{Metodología}

\section{Descripción de tareas, fases y procedimientos}

Aqui irá la descripción de tareas, fases y procedimientos.

\section{Diagrama de Gantt}

Aqui irá el diagrama de Gantt.

\section{Algoritmos}

Aqui irán los algoritmos.

\section{Beneficios que aporta el trabajo}

Aqui irán los beneficios que aporta el trabajo.

\section{Análisis de los resultados}

Aqui irá un análisis de los resultados.


 

\chapter{Aspectos económicos} % Título del capítulo

\label{Economia} % etiqueta \ref{Economia}

\section{Descripción del presupuesto}

Aqui irá las descripción del presupuesto.


\chapter{Conclusiones} % Título del capítulo

\label{Conclusiones} % etiqueta \ref{Conclusiones}

\section{Conclusiones alcanzadas}

Aqui irá las conclusiones alcanzadas. 

\section{Líneas futuras}

Aqui irán las líneas futuras.

%Externalizar por completo los calculos referentes a las RNAs y tener un un modelo completo de RNA computado por un enfoque distribuido. Mejorar la gestion de memoria. Hasta ahora hardcodeada. Posibilidad de usar la DDR (controlador ddr de litex) de la Arty, jtag, debug Usb (issue 38). La unica aproximación se ha hecho utilizando la spi (issue 47) relaizada con exito pero no nos vale (comentario de umarcor)
%Añadir la compilación de software a CI
  

%----------------------------------------------------------------------------------------
%	THESIS CONTENT - APPENDICES
%----------------------------------------------------------------------------------------

\appendix % Cue to tell LaTeX that the following "chapters" are Appendices

% Include the appendices of the thesis as separate files from the Appendices folder
% Uncomment the lines as you write the Appendices


\chapter{Planos} % Título del Anexo

\label{Planos} 

\begin{figure}[h!]
    \vspace*{4cm}
    \centering
    \includegraphics[width=15cm]{Figuras/mult_wfifos.pdf}
    \caption{Plano del multiplicador tipo MULT-B.}
    \label{fig:mult-b}
\end{figure}

\newpage

otro plano


\chapter{\textit{Waveforms}} % Título del Anexo

\label{wave} 

\begin{figure}[h!]
    \centering
    \includegraphics[width=17cm,angle=90]{Figuras/wave-xbus.pdf}
    \caption{\textit{Waveform} resultante del ensayo de \textit{throughput} para NEORV32 + Mult-BP acoplado mediante XBUS.}
    \label{wave:xbus}
\end{figure}

\newpage

\begin{figure}[h!]
    \centering
    \includegraphics[width=22cm,angle=90]{Figuras/wave-cfu.pdf}
    \caption{\textit{Waveform} resultante del ensayo de latencia para NEORV32 + Mult-B acoplado mediante CFU.}
    \label{wave:cfu}
\end{figure}


\chapter{Codigo} % Título del Anexo

\label{Codigo} % Etiqueta \ref{Planos}

Aqui irá el código del diseño.


\chapter{Artículo de congreso} % Título del Anexo

\label{Articulo} % Etiqueta \ref{Planos}

El presente trabajo de investigación ha dado pie a realizar un artículo en el marco del congreso \textit{XXXIX Conference on Design of Circuits and Integrated Systems (DCIS)} que se celebrará del 13 al 15 de Noviembre de 2024 en Catania (Italia).
Cabe destacar que los artículos aceptados para esta conferencia, con al menos un autor registrado y copyright firmado, se publicarán en IEEEXplore.
A este respecto, el siguiente artículo ha sido aceptado y está en espera de ser publicado en dicha base de datos.

\includepdf[pages=-,pagecommand={},templatesize={133mm}{210mm}]{Anexos/paper_73.pdf}
%\includepdf[pages=-, offset=75 -75]{Anexos/paper_73.pdf}


%----------------------------------------------------------------------------------------
%	BIBLIOGRAPHY
%----------------------------------------------------------------------------------------

\printbibliography[heading=bibintoc]

%----------------------------------------------------------------------------------------

\end{document}  
