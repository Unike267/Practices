\chapter{Desarrollo} % Título del capítulo

\label{Desarrollo}

\section{Selección del microcontrolador}

\label{Selec}

Resulta imprescindible que el microcontrolador seleccionado para este proyecto de investigación cumpla con los siguientes requisitos:

%cambiar-listo!
\begin{itemize}
    \item Estar basado en una ISA RISC-V, ya que el contexto del proyecto está orientado a acoplar coprocesadores en núcleos RISC-V.
    \item Estar descrito en el lenguaje de descripción de hardware VHDL, ya que los coprocesadores embebidos para aplicaciones de IA que se pretenden acoplar están descritos en este lenguaje.
    \item Contar con extensión de instrucciones para conectar coprocesadores mediante CFU, además de con soporte para comunicaciones \textit{memory-mapped} e interfaces \textit{stream}, ya que se necesita una variedad de métodos de conexión para realizar la caracterización del rendimiento mediante diferentes modos.
\end{itemize} 

En este sentido, el NEORV32 \cite{gh:neorv32} cumple con todos estos requerimientos. 
Además, en la plataforma de desarrollo colaborativo donde está alojado, cuenta con una comunidad muy activa.
Es por ello que se encuentra bajo una revisión constante de \textit{bugs} (fallos), tanto por parte del autor como de los usuarios.
De esta manera, se asegura en gran medida la correcta operatividad del mismo.
Además, el autor se dedica a realizar actualizaciones periódicas de sus funcionalidades.
Por si fuera poco, tanto el autor como la comunidad tienen una gran disponibilidad para responder dudas sobre temas relacionados con el proyecto, lo que resulta de gran ayuda.
Con respecto a la compilación de lenguajes de alto nivel, el proyecto ofrece \textit{toolchains} precompiladas de RISC-V para GCC.  
Estas herramientas permiten hacer compilación cruzada de C/C++ a instrucciones de RISC-V  en un entorno Linux \cite{gh:neorv32-tool}.
Cabe destacar que también se facilita un contenedor para realizar esta tarea \cite{gh:sim-conatiner}. 
Además, cuenta con un soporte de librerías para compilar funciones software específicas de NEORV32. 
Asimismo, el repositorio ofrece una variedad de ejemplos de aplicación software de todos los recursos con los que cuenta el micro.
En adición a todo lo mencionado, este microcontrolador cuenta con un \textit{datasheet} \cite{neorv32-ds} y una \textit{user guide} \cite{neorv32-ug} realizadas por el autor y actualizadas a la par que el código del proyecto, las cuales destacan por su calidad.
Teniendo en cuenta todas estas consideraciones, el NEORV32 es el procesador seleccionado para este proyecto.

\section{Workflow}

\label{Workf}

\section{Caracterización del rendimiento}

\label{Carac}

\section{Integración de coprocesador para aplicaciones de IA}

\label{Integ}


