\documentclass{article}
\usepackage[margin=0.5in]{geometry}
\usepackage{textcomp}
\usepackage{pgfplots}
\pgfplotsset{width=15cm,compat=1.9}

\begin{document}
% We need layers to draw the block diagram
\pgfdeclarelayer{background}
\pgfsetlayers{background,main}
\begin{tikzpicture}
\begin{axis}[
    %title={Resultados ensayo de latencia},
    ylabel=Ciclos de reloj,
    xtick=data,
    symbolic x coords={CRI, APROX. 7, APROX. 5, APROX. 3},
    enlarge x limits=0.2,
    ybar=7pt,
    bar width=0.5cm,
    nodes near coords,
    ymin=0,
    ymax=295,
    legend style={at={(0.5,-0.1)},anchor=north,legend columns=-1},
    legend entries={MIN , MEDIA , MAX }  % Leyenda de los conjuntos de datos
]
\addplot+[fill=blue!30] coordinates {(CRI,13) (APROX. 7,203) (APROX. 5,157) (APROX. 3,111)};
\addplot+[fill=red!30] coordinates {(CRI,13) (APROX. 7,259.1) (APROX. 5,198.5) (APROX. 3,137.5)};
\addplot+[fill=green!30] coordinates {(CRI,13) (APROX. 7,280) (APROX. 5,210) (APROX. 3,146)};
\end{axis}

\begin{pgfonlayer}{background}
\path[fill=white!20,rounded corners] (-1.25,-2) rectangle ++(15,13.5);
\end{pgfonlayer}
    
\end{tikzpicture}
\end{document}

