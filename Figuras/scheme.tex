\documentclass{article}
\usepackage{tikz}
\usetikzlibrary{shapes,arrows}

\begin{document}
\pagestyle{empty}

% We need layers to draw the block diagram
\pgfdeclarelayer{background}
\pgfdeclarelayer{foreground}
\pgfsetlayers{background,main,foreground}

%Definicion tamaños rectangulo (ancho,alto)
\def\CPU{5.15,2}
\def\medio{3.4,0.9}
\def\peque{1.65,0.9}
\def\cfs{7,0.9}
\def\cfu{3.04,0.9}
\def\cntr{1.52,0.9}
\def\acc{1.75,0.9}
%Separación vertical
\def\SepVer{0.1}
%Separación horizontal
\def\SepHor{0.5}
%Separacion horizontal fuera
\def\SepHorFU{2.25}
\def\AccVerP{1.7}
\def\AccVerS{-0.45}
\def\AccVerT{-2.6}
\def\AccVerC{-4.75}
\def\xAcc{8.5}

\begin{tikzpicture}
\node (center) {};
%Rectangulo CPU (coordenadas) rectangle ++(tamaño)
\draw[fill=orange!20,dotted] (1.5-0.15+\SepHor,0) rectangle ++(\CPU);
%Nombre del bloque CPU
\path (center)+(1.5+\SepHor+1.25,1.6) node (name) {NEORV32 CPU};
\path (center)+(1.5+\SepHor+3.9,1.6) node (name) {RISC-V};
%Rectangulo CFU
\draw[fill=blue!20] (1.85+\SepVer+\SepVer+1.52+\SepVer+\SepVer,\SepVer+\SepVer) rectangle ++(\cfu);
%Nombre del bloque CFU
\path (center)+(2+\SepVer+\SepVer+1.52+\SepVer+\SepVer+1.465-0.12,\SepVer+\SepVer+0.45) node (name) {Zxcfu};
%Rectangulo CNTR
\draw[fill=blue!20] (1.85+\SepVer+\SepVer,\SepVer+\SepVer) rectangle ++(\cntr);
%Nombre del bloque CNTR
\path (center)+(2+\SepVer+\SepVer+0.73-0.12,\SepVer+\SepVer+0.45) node (name) {Zicntr};
%Rectangulo OCD
\draw[fill=orange!30] (0,1+\SepVer) rectangle ++(\peque);
%Nombre del bloque OCD
\path (center)+(0.75+0.075,1.55) node (name) {OCD};
%Rectangulo UART0
\draw[fill=orange!30] (0,0) rectangle ++(\peque);
%Nombre del bloque UART0
\path (center)+(0.75+0.075,0.45) node (name) {UART0};
%Rectangulo Bus System
\draw[fill=orange!20] (0,-\SepVer-1) rectangle ++(\medio);
%Nombre del bloque Bus System
\path (center)+(1.625+0.075,-0.65) node (name) {Bus System};
%Rectangulo SYSINFO
\draw[fill=orange!20] (3.25-0.15+\SepHor,-\SepVer-1) rectangle ++(\medio);
%Nombre del bloque SYSINFO
\path (center)+(3.75+1.625-0.075,-0.65) node (name) {SYSINFO};
%Rectangulo DMEM
\draw[fill=orange!20] (0,-\SepVer-2.1) rectangle ++(\medio);
%Nombre del bloque DMEM
\path (center)+(1.625+0.075,-1.75) node (name) {DMEM};
%Rectangulo IMEM
\draw[fill=orange!20] (0,-\SepVer-3.2) rectangle ++(\medio);
%Nombre del bloque IMEM
\path (center)+(1.625+0.075,-2.85) node (name) {IMEM};
%Rectangulo SLINK
\draw[fill=gray!20] (3.25-0.15+\SepHor,-\SepVer-2.1) rectangle ++(\medio);
%Nombre del bloque SLINK
\path (center)+(3.75+1.625-0.075,-1.75) node (name) {SLINK};
%Rectangulo XBUS
\draw[fill=gray!20] (5-0.15+\SepHor,-\SepVer-3.2) rectangle ++(\peque);
%Nombre del bloque XBUS
\path (center)+(5.725+\SepHor-0.075,-2.85) node (name) {XBUS};
%Rectangulo XCACHE
\draw[fill=gray!50] (3.25-0.15+\SepHor,-\SepVer-3.2) rectangle ++(\peque);
%Nombre del bloque XCACHE
\path (center)+(4+\SepHor-0.075,-2.85) node (name) {CACHE};
%Union
\draw[fill=black] (3.25-0.15+\SepHor+1.65,-\SepVer-3.2+0.4) rectangle ++(0.1,0.1);
%Rectangulo CFS
\draw[fill=gray!20] (0,-\SepVer-4.3) rectangle ++(\cfs);
%Nombre del bloque CFS
\path (center)+(3.5,-3.95) node (name) {CFS};
%JTAG name
\path (center)+(-\SepHorFU,1.55) node (jtag) {JTAG};
%Flecha JTAG
\draw[->] (jtag) to (-0.5,1.55);
%Rectangulo PC
\draw[fill=red!30] (-0.75-\SepHorFU,0) rectangle ++(\peque);
%Nombre del bloque PC
\path (center)+(-\SepHorFU+0.075,0.45) node (name) {PC};
%Flecha PC
\draw[<->] (0.75-\SepHorFU+0.15,0.45) to (-0.5,0.45);

%Nombre Custom
\path (center)+(9.6,4) node (name) {Aceleradores};
\path (center)+(9.6,3.5) node (name) {Personalizados};
%acceler1
\draw[fill=green!20,opacity=0.4] (\xAcc+0.2+0.2,\AccVerP+0.2+0.2) rectangle ++(\acc);
\draw[fill=green!20,opacity=0.6] (\xAcc+0.2,\AccVerP+0.2) rectangle ++(\acc);
\draw[fill=green!20] (\xAcc,\AccVerP) rectangle ++(\acc);
%acceler2
\draw[fill=green!20,opacity=0.4] (\xAcc+0.2+0.2,\AccVerS+0.2+0.2) rectangle ++(\acc);
\draw[fill=green!20,opacity=0.6] (\xAcc+0.2,\AccVerS+0.2) rectangle ++(\acc);
\draw[fill=green!20] (\xAcc,\AccVerS) rectangle ++(\acc);
%acceler3
\draw[fill=green!20,opacity=0.4] (\xAcc+0.2+0.2,\AccVerT+0.2+0.2) rectangle ++(\acc);
\draw[fill=green!20,opacity=0.6] (\xAcc+0.2,\AccVerT+0.2) rectangle ++(\acc);
\draw[fill=green!20] (\xAcc,\AccVerT) rectangle ++(\acc);
%acceler4
\draw[fill=green!20,opacity=0.4] (\xAcc+0.2+0.2,\AccVerC+0.2+0.2) rectangle ++(\acc);
\draw[fill=green!20,opacity=0.6] (\xAcc+0.2,\AccVerC+0.2) rectangle ++(\acc);
\draw[fill=green!20] (\xAcc,\AccVerC) rectangle ++(\acc);
%Flecha acceler1
\draw[<-] (7.5,\SepVer+\SepVer+0.45) to (8-0.125,\SepVer+\SepVer+0.45);
\draw (8-0.125,\SepVer+\SepVer+0.45) -- (8-0.125,\AccVerP+0.45);
\draw[->] (8-0.125,\AccVerP+0.45) to (\xAcc-0.25,\AccVerP+0.45);
%Flecha acceler2
\draw[<-] (7.5,-1.75) to (8-0.125,-1.75);
\draw (8-0.125,-1.75) -- (8-0.125,\AccVerS+0.45);
\draw[->] (8-0.125,\AccVerS+0.45) to (\xAcc-0.25,\AccVerS+0.45);
%Flecha acceler3
\draw[<-] (7.5,-2.85) to (8-0.125,-2.85);
\draw (8-0.125,-2.85) -- (8-0.125,\AccVerT+0.45);
\draw[->] (8-0.125,\AccVerT+0.45) to (\xAcc-0.25,\AccVerT+0.45);
%Flecha acceler4
\draw[<-] (7.5,-3.95) to (8-0.125,-3.95);
\draw (8-0.125,-3.95) -- (8-0.125,\AccVerC+0.45);
\draw[->] (8-0.125,\AccVerC+0.45) to (\xAcc-0.25,\AccVerC+0.45);

%Nombre procesador
\path (center)+(1.55,2.5) node (name) {Procesador NEORV32};

%Nombre del bloque
\path (center)+(1.5+0.1,3.75) node (name) {Diseño SoC Personalizado};

\begin{pgfonlayer}{background}
    \path [fill=white!20] (-3.3,-5.5) rectangle ++(14.85,10.25);
    \path [fill=white!20,rounded corners, draw=black!50, dashed] (-0.85,-5.25) rectangle ++(12.1,9.8);
    \path [fill=green!10,opacity=0.4,rounded corners, draw=black!50, dashed] (\xAcc-0.25,\AccVerC-0.2) rectangle ++(0.2+0.2+1.75+0.5,7.75+1.5);
    \path [fill=yellow!20,rounded corners, draw=black!50, dashed] (-0.5,-4.75) rectangle ++(8,7.75);
\end{pgfonlayer}
    
\end{tikzpicture}
\end{document}

